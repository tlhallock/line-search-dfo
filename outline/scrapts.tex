

%This paper will discuss research in derivative free optimization (DFO).
%It begins with an introduction to derivative free optimization \color{red}supplemented by some of the recent advancements\color{black}.
%It then details several approaches for optimization over thin feasible regions.
%The focus is on model-based trust region algorithms for local search within constrained derivative free optimization.
%Specifically, we are interested in problems with thin, \color{red}hidden\color{black} constraints.
%Narrow constraints introduce numerical instability and make it hard to to model the relavent functions.


%Add section on importance of derivatives


For box constraints there are these softwares
SPBOX, PSwarm, all NLopt algorithms except COBYLA (BOBYQA, NEWUOA, PRAXIS, Sbplx).



This means that we have as many points for which we know the constaints as points for which we know the objective.
This creates an interesting situation within model-based approaches which use different orders of models for the objective than the constraints.
The algorithm designer must decide how to choose a subset of these points, use a higher order model, or fit an overdetermined model.

If the constraints can be evaluated at points outside the feasible region, the constraints are called \emph{relaxable} constraints.
Some problems additionally contain ``hidden" constraints which are not explicit in the model but merely result in a notification that the objective could not be evaluated at the requested point.
For example, this can arise when simulation software fails.
This may mean that it is not possible to tell how close to a ``hidden" constraint an iterate lies.


Another area that received attention recently is that of imposing structure on $f$, and $c$.
For example, a method called Practical Optimization Using No Derivatives for sums of Squares
is developed within \color{red}citation\color{black} %\cite{DUMMY:leastsquares} 
when $f$ takes the form of a least squares error
$f(x) = \frac 1 2 \|F(x)\|^2$ over bound constraints where $F$ is a nonlinear, vector valued function.


This program becomes most difficult when the all functions are simulated, including the constraints.




Within derivative free optimization, we focus on constrained optimization.


This means that 




%Within simulated constraints, there are still several interesting categories.
%Of particular importatance to this paper is that of 


%\begin{red}This means that each call to the objective gives values of the constraints as well, and vice versa.\end{red}

% This produces $c(x, S(x))=c(S(x))$.




%One of the primary goals within derivative free optimization is to solve programs while avoiding as many of these expensive function evaluations as possible.






%The number of function evaluations can grow when approximating derivatives that are not given explicitly.










If available, they should be used. 
% However, they are not always available.


\subsubsection{Problem formulations}

This proposal aims to develop derivative-free algorithms for solving constrained nonlinear optimization problems of the form

We will introduce several different forms this problem can take noting those which we will discuss algorithms for.
For a more comprehensive characterization of the types constraints, consult \color{red}citation\color{black} %\cite{DUMMY:typesofconstraints}.


\subsubsection{Where Problems come from}

%\color{red}
%laziness and parameter tuning were circled.


%There are a growing number of applications for DFO. 
%For example, derivative free methods can be useful when the objective is the result of a simulation that does not admit automatic differentiation.
%As the popularity of complicated simulations increase, so does the demand for optimizing over black box software codes.
% which may be copyrighted.
%Derivative free optimization has also been a popular method for parameter tuning, as simulations may have several parameters with unidentified relationships to their %output.

%\color{black}




%Sometimes user laziness can preclude derivative information.
Even when it would be possible to compute derivative information, it may be prohibatively time consuming.




%Many DFO methods simply let $f(x,S(x)) = S(x)$.



%\subsubsection{Importance of Derivatives}

%The lack of derivative information means that DFO methods are at a disadvantage when compared to their counterparts in nonlinear optimization.
%First and second derivative information is explicit in algorithms with quadratic convergence such as Newton's method.
%They are also present in conditions for convergence results such as Wolf's, Armijo or Goldstien for line search methods.
%Additionally, stopping criteria usually involve a criticality test involving derivatives.
%When derivatives are known, they should be used.
%% For this reason, it is desirable for $n$, the dimension of $x$, to be small.




%\subsubsection{Noise. Deterministic versus stochastic}
%
%One branch of DFO is concerned with noisy evaluations of $S$.
%Noisy functions can be categorized as either deterministic or stochastic.
%Roundoff error, truncation error and finite termination errors can result in what is called deterministic noise.
%This means that although a function is not evaluated accurately, the error will not change across multiple function calls with the same input.
%On the other hand, stochastic noise means that each point in the domain is associated with a distribution of possible values $S$ may return.
%In this paper we assume $S$ is not noisy.


%A trend within derivative free optimization is the permission for larger tolerances within solutions.
%Their functions are frequently expensive to evaluate, so we can only ask for a small number of significant figures.
%This implies slightly less concern for asymptotic convergence rates.

\subsubsection{Automatic Differentiation}

When $S$ is the result of a simulation for which the source code is available, one convenient approach is to perform automatic differentiation.
Although derivatives of complicated expressions resulting from code structure are difficult to work with on paper, the rules of differentiation can be applied algorithmically.
However, the nature of the code or problem can make this very difficult: for example with combinatorial problems that rely heavily on if statements.

\subsubsection{Direct search}


Another approach is to use direct search methods that do not explicitly estimate derivative information but evaluate the objective on a pattern or other structure to find a descent direction.
Examples of this include Coordinate descent, implicit filtering and other pattern based search methods.
One of the most popular direct search method is Nelder Mead, which is implemented in fminsearch in Matlab.
It remains popular although it is proven to not converge in pathological cases unless modifications are made.

These methods can be robust in that they are insensitive to scaling and often converge to a local minimum even when assumptions such as smoothness or continuity are violated.
However, they ignore potentially helpful information because they do not use derivative information provided through the function evaluations.
This means that they can also lack fast convergence rates.

%\color{red}
%(0th derivative)
%\color{black}

\subsubsection{Finite difference methods}

Finite difference methods can be used to approximate the derivative of a function $f$.
One common approximation called the symmetric difference is given by $\nabla f(x) \approx (\frac{f(x+he_i) - f(x-he_i)}{2h})_i$ for some small $h$ where $e_i = (0,\ldots, 0, 1, 0, \ldots, 0)^T \quad \forall \; 1 \le i \le n$ is the unit vector with $1$ in its $i$th component.
%This may work well, but can have issues with unlucky iterates (CITE).
However, the number of function evaluations tends to grow large with the dimension and number of iterations the algorithm performs.
This is because derivative information is only gathered near the current iterate when $h$ is small, which is required for accurate derivative calculations.
Because of the large number of function evaluations required for finite difference schemes, it may preferable to spread sample points out over the entire region where we may expect to step.
Also, function evaluations may be subject to noise, making the finite difference approximations of derivative problematic.

\subsubsection{Model based methods}
%\color{red}
%Kriging seems to be another popular model function, but I usually see it used within global optimization.
%\color{black}



\subsubsection{Trust region versus Linesearch} \label{linevsmodel}

Within derivative free optimization, we can ensure the accuracy of our model function by sampling points over a small enough trust region.
However, reducing the trust region implies more points must be evaluated.
Linesearch methods rely on the the ability to calculate a descent direction that will be accurate in a small enough region around the current iterate:
small enough that the trust region must be reduced to ensure the model's accuracy.

This means trust region framework fits into derivative free optimization more naturally than line search methods.
Not only do the trust regions arise naturally, but many line search algorithms exploit how much easier it is to find a descent direction than solve a trust region subproblem.
However, in derivative free optimization, solving a costly trust region subproblem is acceptable if it allows us to avoid even more expensive function evaluations.


% We began with a line search filter method, however we found that this had several drawbacks:
% \begin{itemize}
% \item Line search algorithms exploit how much easier finding a descent direction is than solving the trust region subproblem, but this saved computation is not as useful in DFO as here it is function evaluations that we avoid
% \item As the algorithm backtracks on the step length, the trust region must be reduced, as the models are accurate over a region rather than at a single point
% \end{itemize}



\subsubsection{Literature Background}

\paragraph{Derivative free methods}

%Within  \cite{DUMMY:intro_book} derivative-free methods are developed in detail.
%This is the first text book devoted to derivative free optimization.
%It contains a good explanation of ensuring geometry of the current set with poisedness for unconstrained problems and also covers other derivative-free methods including direct-search and line search.

%A good review of derivative free algorithms and software libraries can be found in \cite{DUMMY:review}.
%This compares several software libraries, and reviews the development of derivative free optimization since it started.
%Another recent review can be found in \cite{DUMMY:review2}.

%Within \cite{DUMMY:linesearch_global} and \cite{DUMMY:linesearch_local} Biegler uses a filter method to ensure global convergence within a line search framework.

\section{Experiments}












\paragraph{Basis functions}

The choice of model functions $\phi_i$ can have some affect on the convergence rate, as Powell showed in \color{red}citation\color{black} % \cite{DUMMY:PowellRadialBasis}.
One common choice of basis functions is the Lagrange polynomials, in which we select polynomials satisfying $\phi_{i}(y^j) = \delta_{ij}$, the kroneker delta function.
This reduces the matrix within \ref{reg} to an identity matrix.
Lagrange polynomials of order $p$ can be computed by starting with the monomial basis $\prod_{i=1}^{n} x_i^{n_i}$ for all choices of $0 \le n_i \le p$ with $\sum_{i=1}^n n_i \le p$ and inverting the corresponding Vandermonde matrix.

Newton's Fundamental polynomials are also used, and follow a similar pattern.
However, they maintain different orders of polynomials within the basis:
a single constant value, a set of $n+1$ linear polynomials,
$n + {n \choose 2}$ quadratic functions, and more for higher order polynomials.
Radial basis functions may have some intuitive advantage because the algorithm makes claims about the accuracy of the function over a trust region.


Model functions are usually chosen to be fully linear or fully quadratic: terms describing how the model's error grows as a function of the trust region radius.



