
\newcommand{\gik}{{g^{(i, k)}}}
\newcommand{\hgik}{{{\hat g}^{(i, k)}}}
\newcommand{\zik}{{z^{(i, k)}}}
\newcommand{\fik}{{\mathcal F_{i, k}}}
\newcommand{\iik}{{\mathcal I_{k}}}
\newcommand{\uk}{{u^{(k)}}}
\newcommand{\huk}{{{\hat u}^{(k)}}}
\newcommand{\ask}{{\alpha^{(\star, k)}}}
\newcommand{\bsk}{{\beta^{(\star, k)}}}
\newcommand{\bs}{{\beta^{\star}}}
\newcommand{\fcko}{{\mathcal {F}^{\text{outer}}_k}}
\newcommand{\fcki}{{\mathcal {F}^{\text{inner}}_k}}
\newcommand{\rn}{{\mathbb R^{n}}}
\newcommand{\xsk}{{x^{(\star, k)}}}
\newcommand{\bxk}{{\bar{x}^{(k)}}}
\newcommand{\wik}{{w^{(i, k)}}}
\newcommand{\f}{{\mathcal F}}
\newcommand{\lgi}{{L_{g, i}}}

\subsection{Notation}
Then for all $1 \le i \le m$, $k = 1, \ldots$, define the following whenever they exist:
\begin{align}
\alpha, \beta \in (0, 1) \label{def_alpha_beta} \\
\|\nabla c_i(x) - \nabla c_i(y)\| \le \lgi \|x - y\| \label{def_lipshitz} \\
\f = \{x \in \rn | \quad c_i(x) \le 0\quad \forall 1 \le i \le m \} \label{def_feasible} \\
\gik = \nabla m_{c_i}(\xk) \label{def_g} \\
\gik = \nabla c_i(\xk) \label{def_cg} \\
\hgik = \frac{\gik}{\|\gik\|} \label{def_gk} \\
\zik = \xk - \frac{m_{c_i}(\xk)}{\|\gik\|}\hgik \label{def_z} \\
\wik = \xk + \left(1 - \alpha\dk^{\frac 1 2}\right)\left(\zik - \xk\right) \label{def_w} \\
\fik = \left\{x \in \rn | x = \wik + t s,t > 0, \|s\| = 1, -s^T\hgik \ge \beta \dk^{\frac 1 2 }\right\} \label{def_f} \\
\iik = \{1\le i \le m | \zik \in B_{\infty}(\xk, \dk) \} \label{def_i} \\
\uk = -\nabla \hat m_{c_{\mathcal I}}(\xk)^{\dagger} e \label{def_u} \\
\huk = \frac {\uk} {\| \uk\|} \label{def_hu} \\
\theta^{\text{min}}_k = \min_{i \in \mathcal I} (-\hgik)^T \huk \label{def_theta_k_min} \\
\theta^{\text{max}}_k = \max_{i \in \mathcal I} (-\hgik)^T \huk \label{def_theta_k_max} \\
\theta^{\text{min}} = \liminf_{k\to\infty} \theta^{\text{min}}_k \label{def_theta_min} \\
\theta^{\text{max}} = \limsup_{k\to\infty} \theta^{\text{max}}_k \label{def_theta_max} \\
\bs = \sqrt{ 1 - (\theta^{\text{max}}_k)^2} \label{def_bs} \\
\bsk = \beta\dk^{\frac 1 2} \theta^{\text{max}}_k + \sqrt{(1 - \dk\beta^2)\left(1 - (\theta^{\text{max}}_k) ^2\right)} \label{def_bs_k} \\
\fcki = \{x \in \rn | \quad x = \xk + ts, t > 0, \|s\| = 1, s^T\huk \ge \bsk \} \label{feasible_cone} \\
0 < g_{\text{low}} \le \|\gik\| \le g_{\text{hi}} \label{def_g_bounds} \\
0 < h_{\text{low}} \le \|\nabla^2m_{c_i}(x)\| \le h_{\text{hi}} \label{def_h_bounds} \\
f_e(\delta, r, \theta; x) = (x - \delta e_1)^T\bigg(\begin{bmatrix}
1 & \boldsymbol0^T \\
\boldsymbol 0 & \frac{\theta^2}{1 - \theta^2} \boldsymbol I \\
\end{bmatrix}\bigg)(x - \delta e_1) - \frac 1 2 r^2 \label{def_ellipse_function} \\
R = 2\frac{(e_1 + \huk)(e_1 + \huk)^T}{(e_1 + \huk)^T(e_1 + \huk)} - \boldsymbol I \label{def_rotation} \\
E_k = \{x \in \rn | f_e\left(\frac 1 2 \dk, \frac 1 2 \dk, \bsk; R(x - \xk)\right) \le 0\} \label{def_ellipse_k} \\
\nabla c_i(\xk) = \nabla m_{c_i}(\xk) + \epsilon_{g}\dk^2\nu \label{def_lambda_poised} \\
\inf_{k}\min\{\sigma(\nabla c_{\mathcal I_k}(\xk)) \; | \; \sigma(\nabla c_{\mathcal I_k}(\xk)) > 0 \} \ge \epsilon_{\sigma} > 0\\
\end{align}



% E_k = \{x \in \rn | f_e\left(\frac 1 2 \dk, \frac 1 2 \dk, \bsk; R(x - \xk)\right) \le 0\} \label{def_lambda_poised} \\

% \fcko = \{x \in \rn | \quad x = \xsk + \ask\dk^{\frac 1 2}\huk + ts, t > 0, \|s\| = 1, s^T\huk \ge \bsk \} \\
% \ask = \frac{1 + \alpha}{\theta^{\text{min}}_k} \\
% \bxk = \xsk + (\huk)^T(\xk - \xsk)\huk\\
% m_{\mathcal I}(\xk) + \nabla m_{\mathcal I}(\xk)^T (\xsk - \xk) = 0 \\

\subsection{Derivation of feasible ellipse}


\begin{theorem}
Let $\theta^{\text{min}}$ be as defined as in \cref{def_theta_min}.
Then $\theta^{\text{min}} > 0.$
\end{theorem}

\begin{proof}
We assume that $\nabla c_{\mathcal I_k}(\xk)$ has full row and column rank.
Let $n_k$ be the number of rows in $\nabla c_{\mathcal I_k}(\xk)$, and $d = \min \{ n, n_k \}$.
Give $\nabla m_{c_{\mathcal I_k}}(\xk)$ and $\nabla c_{\mathcal I_k}(\xk)$ their singular value decompositions
\begin{align*}
\nabla m_{c_{\mathcal I_k}}(\xk) = U\Sigma V^T, \quad
\nabla c_{\mathcal I_k}(\xk) = \hat U \hat \Sigma \hat V^T
\end{align*}
where $U, \hat U$ are orthonormal $n_k \times d$ matrices; $\Sigma, \hat \Sigma$ are diagonal $d \times d$ matrices; and $V, \hat V$ are orthonormal $d \times n$ matrices.
Note that because of \cref{def_lambda_poised}, if $\|\cdot\|_{\mathcal F}$ is the Frebenious norm, then
\begin{align*}
\|\nabla c(\xk) - \nabla m_{c_{\mathcal I_k}}(\xk)\|_{\mathcal F}^2 \le n_k \left(\epsilon_g \dk^2\right)^2
\Longrightarrow \|\nabla c(\xk) - \nabla m_{c_{\mathcal I_k}}(\xk)\|_{\mathcal F} \le \sqrt{n_k} \epsilon_g \dk^2.
\end{align*}
Each element $\hat \Sigma$ is greater than $\epsilon_{\sigma}$, so because $\dk \to 0$ we have that for sufficiently large $k$, each element of $s$ of $\Sigma$ satisfies $s > \frac 1 2 \epsilon_{\sigma}$.

First, we compute
\begin{align*}
\nabla \hat m_{c_{\mathcal I_k} }(\xk) \uk = -\nabla\hat m_{c_{\mathcal I_k} }(\xk) \nabla\hat m_{c_{\mathcal I_k} }(\xk)^{\dagger} e
= -(U\Sigma V^T)(V\Sigma^{\dagger} U^T)e
= -e 
\end{align*}
and
\begin{align*}
\|\uk\| = \|V\Sigma^{\dagger} U^Te\|  
\le \frac 1 {\sigma_{\text{min}}\left(\nabla \hat m_{c_{\mathcal I_k}}(\xk)\right)}.
\end{align*}

Then we can see that
\begin{align*}
(-\gik)^T\huk = -\frac{\|\gik\|}{\|\uk\|}(\hgik)^T\uk = \frac{\|\gik\|}{\|\uk\|} \ge g_{\text{low}} \sigma_{\text{min}}\left(\nabla \hat m_{c_{\mathcal I_k}}(\xk)\right) \ge \frac 1 2 g_{\text{low}}\epsilon_{\sigma}. \\
\end{align*}

% -A^T(AA^T)^{-1}Ae = \\


\end{proof}




\begin{theorem}
\label{cone_subset_cone}
Given $u^1, u^2 \in \rn$, $\|u^1\| = \|u^2\|= 1$, $\beta >0$, with ${u^1}^Tu^2 \ge \beta$ define
\begin{align*}
B = \{x\in\rn | {u^2}^Tx \ge \beta\|x\|\}, \quad
S = \left\{x\in\rn \bigg| {u^1}^Tx \ge \left(\beta {u^1}^Tu^2 + \sqrt{(1 - \beta^2)\left(1 - ({u^2}^Tu^1)^2\right)}\right)\|x\| \right\}. 
\end{align*}
Then, $S \subseteq B$.
\end{theorem}


\begin{proof}
Let 
\begin{align*}
x^{\star} = \beta u^2 + \sqrt{\frac{1 - \beta^2}{1 - ({u^2}^Tu^1)^2}} (u^1 - {u^2}^Tu^1 u^2 )
\end{align*} and for a contradiction, let $y \in \rn$ be such that $y \not \in B$ and $y \in S$ and define $\hat y = \frac{y}{\|y\|}$.
Then
\begin{align*}
\hat y \in \left\{x \in \rn | {u^2}^Tx < \beta, {u^1}^Tx \ge \beta {u^1}^Tu^2 + \sqrt{(1 - \beta^2)\left(1 - ({u^2}^Tu^1)^2\right)} \right\}.
\end{align*}

Note that
\begin{align}
{u^1}^Tx^{\star} &=& {u^1}^T\left(\beta u^2 + \sqrt{\frac{1 - \beta^2}{1 - ({u^2}^Tu^1)^2}} (u^1 - {u^2}^Tu^1 u^2 )\right) = 
\beta {u^1}^Tu^2 + \sqrt{(1 - \beta^2)\left(1 - ({u^2}^Tu^1)^2\right)} \\
{u^2}^Tx^{\star} &=& {u^2}^T\left(\beta u^2 + \sqrt{\frac{1 - \beta^2}{1 - ({u^2}^Tu^1)^2}} (u^1 - {u^2}^Tu^1 u^2 )\right) = 
\beta + \sqrt{\frac{1 - \beta^2}{1 - ({u^2}^Tu^1)^2}} ({u^2}^Tu^1 - {u^2}^Tu^1 ) = \beta.
\end{align}
That means
\begin{align}
{u^1}^T\hat y = {u^1}^T\left(x^{\star} + \hat y - x^{\star}\right) = \beta {u^1}^Tu^2 + \sqrt{(1 - \beta^2)\left(1 - ({u^2}^Tu^1)^2\right)} + {u^1}^T\left(\hat y - x^{\star}\right) \\
\ge \beta {u^1}^Tu^2 + \sqrt{(1 - \beta^2)\left(1 - ({u^2}^Tu^1)^2\right)} 
\Longrightarrow {u^1}^T\left(\hat y - x^{\star}\right) \ge 0 \\
{u^2}^T\hat y = {u^2}^T\left(x^{\star} + \hat y - x^{\star}\right) = \beta + {u^2}^T\left(\hat y - x^{\star}\right) < \beta
\Longrightarrow {u^2}^T\left(\hat y - x^{\star}\right) < 0. \label{the_difference_is_nonzero}
\end{align}

From these two equations, we not only know that $\left(\hat y - x^{\star}\right) \ne 0$, but
\begin{align*}
{\left(\hat y - x^{\star}\right)}^Tx^{\star} = 
\left(\beta {\left(\hat y - x^{\star}\right)}^Tu^2 + \sqrt{\frac{1 - \beta^2}{1 - ({u^2}^Tu^1)^2}} \left({\left(\hat y - x^{\star}\right)}^Tu^1 - {u^2}^Tu^1 {\left(\hat y - x^{\star}\right)}^Tu^2 \right)\right) > 0
\end{align*}
as ${u^1}^Tu^2 \ge \beta \Longrightarrow {u^2}^Tu^1\sqrt{\frac{1 - \beta^2}{1 - ({u^2}^Tu^1)^2}} \ge \beta$.
However, this is a contradiction as
\begin{align*}
1 = \|\hat y\| = \|x^{\star} + \hat y - x^{\star}\| > \|x^{\star}\| = 1
\end{align*}
and there is no such $y$.
Thus, any $y \in\rn$ with $y \in S$ must also have $y \in B$.
\end{proof}






\begin{theorem}
If 
\begin{align}
z^{(i, k)} \in B_{\infty}(\xk, \dk)) \label{z_is_active} \\
M \ge \sup_{x \in B_{\infty}(\xk, \dk)} \frac 1 2 \nabla^2 c_i(x) \label{m_bounds} \\
\text{and} \quad \dk \le \min\left\{
1,
\left(\frac{\alpha g_{\text{low}}}{M \sqrt{n} + \epsilon_g}\right)^2,
\left(\frac{\beta}{2\epsilon_{g}}g_{\text{low}}\right)^{\frac 2 3},
\left[\frac {g_{\text{low}} \beta} {2M\sqrt{n}\left(1 + \frac {\lgi} M \right)} \right]^2
\right\}, \label{delta_is_small_enough}
\end{align} then
\begin{align*}
c_i(x) \le 0 \quad \forall x \in \fik \cap B_{\infty}(\xk, \dk).
\end{align*}

\end{theorem}

\begin{proof}

Let 
\begin{align}
y = \wik + ts \in \fik \cap B_{\infty}(\xk, \dk) \label{t_is_bounded}
\end{align}
with $t > 0, \|s\| = 1, -s^T\hgik \ge \beta \dk^{\frac 1 2}$.
Because the model $m_{c_i}$ is fully quadratic, we know that there exists a $\nu\in\rn$ such that \cref{def_lambda_poised} holds:
\begin{align*}
\nabla c_i(\xk) = \nabla m_{c_i}(\xk) + \epsilon_{g}\dk^2\nu.
\end{align*}

First, we will show that $c(\wik) \le 0$.
For simplicity, we use \cref{def_cg}, \cref{def_gk}, \cref{def_z}, and \cref{def_w} to compute
\begin{align}
\wik - \xk & = & \xk + \left(1 - \alpha \dk^{\frac 1 2 }\right)\left(\zik - \xk\right) - \xk 
= \left(1 - \alpha \dk^{\frac 1 2 }\right)\left(\zik - \xk\right) \\
& = &  \left(1 - \alpha \dk^{\frac 1 2 }\right)\left(\xk - \frac{c_i(\xk)}{\|\gik\|^2}\gik - \xk\right) 
= \left(1 - \alpha \dk^{\frac 1 2 }\right)\frac{-c_i(\xk)}{\|\gik\|^2}\gik. \label{simple_computation}
\end{align}

Also, note by \cref{delta_is_small_enough}, \cref{def_alpha_beta} that both
\begin{align*}
\dk \le \left(\frac{\alpha \|\gik\|}{M \sqrt{n} + \epsilon_g}\right)^2 \Longrightarrow 
M \sqrt{n}\dk^{\frac 1 2} + \epsilon_g \dk^{\frac 1 2} \le \alpha \|\gik\| \quad \text{and} \quad
0 < 1 - \alpha \dk^{\frac 3 2 } < 1.
\end{align*}
We can combine these and \cref{z_is_active} to find 
\begin{align*}
M \sqrt{n}\dk^{\frac 1 2}\left(1 - \alpha \dk^{\frac 3 2 }\right)^2  + \epsilon_g \dk^{\frac 3 2} \left(1 - \alpha \dk^{\frac 3 2 }\right) \le M \sqrt{n}\dk^{\frac 1 2} + \epsilon_g \dk^{\frac 1 2} & \le & \alpha \|\gik\| \\
\Longrightarrow -\alpha \dk^{\frac 1 2 }\|\gik\| + M \left(1 - \alpha \dk^{\frac 3 2 }\right)^2 \sqrt{n}\dk+ \epsilon_g \dk^2 \left(1 - \alpha \dk^{\frac 3 2 }\right) & \le &  0 \\
\Longrightarrow -\alpha \dk^{\frac 1 2 }\|\gik\|\left(-\frac{c_i(\xk)}{\|\gik\|}\right) + M \left(1 - \alpha \dk^{\frac 3 2 }\right)^2 \left(-\frac{c_i(\xk)}{\|\gik\|}\right)^2+ \epsilon_g \dk^2 \left(1 - \alpha \dk^{\frac 3 2 }\right)\left(-\frac{c_i(\xk)}{\|\gik\|}\right) & \le& 0.
\end{align*}

However, we also know by \cref{m_bounds}, \cref{simple_computation} and the Mean Value Theorem, that
\begin{align*}
0 &\ge& \alpha \dk^{\frac 1 2 } c_i(\xk) + M \frac {c_i(\xk)^2}{\|\gik\|^2}\left(1 - \alpha \dk^{\frac 1 2 }\right)^2 + \epsilon_g \dk^2 \left(1 - \alpha \dk^{\frac 1 2 }\right)\frac{-c_i(\xk)}{\|\gik\|} & \\
&\ge& \alpha \dk^{\frac 1 2 } c_i(\xk) + M \frac {c_i(\xk)^2}{\|\gik\|^2}\left(1 - \alpha \dk^{\frac 1 2 }\right)^2 + \epsilon_g \dk^2 \left(1 - \alpha \dk^{\frac 1 2 }\right)\frac{-c_i(\xk)}{\|\gik\|^2}\nu^T\gik & \\
&=& c_i(\xk)\left[1 - \left(1 - \alpha \dk^{\frac 1 2 }\right)\right] + M \frac {c_i(\xk)^2}{\|\gik\|^2}\left(1 - \alpha \dk^{\frac 1 2 }\right)^2 + \epsilon_g \dk^2\nu^T \left(\wik - \xk\right) & \\
&& + \epsilon_g \dk^2\nu^T \left(\wik - \xk\right) & \\
&=& c_i(\xk) + \left(\gik\right)^T\left(1 - \alpha \dk^{\frac 1 2 }\right)\frac{-c_i(\xk)}{\|\gik\|^2}\gik + M \left\|\left(1 - \alpha \dk^{\frac 1 2 }\right)\frac{-c_i(\xk)}{\|\gik\|^2}\gik\right\|^2 & \\
&=& c_i(\xk) + \left(\gik\right)^T\left(\wik - \xk\right)+ M \left\|\wik - \xk\right\|^2  + \epsilon_g \dk^2\nu^T \left(\wik - \xk\right) & \\
 & = & c_i(\xk) + \nabla c_i(\xk)^T\left(\wik - \xk \right) + M \left\|\wik - \xk\right\|^2 \ge c_i(\wik). &
\end{align*}

% \begin{align*}
% \alpha \dk^{\frac 3 2 } c_i(\xk) + M \frac {c_i(\xk)^2}{\|\gik\|^2}\left(1 - \alpha \dk^{\frac 3 2 }\right)^2 + \epsilon_g \dk^2 \left(1 - \alpha \dk^{\frac 3 2 }\right)\frac{-c_i(\xk)}{\|\gik\|} \le 0\\
% \alpha \dk^{\frac 3 2 } c_i(\xk) + M \dk^2\left(1 - \alpha \dk^{\frac 3 2 }\right)^2 + \epsilon_g \dk^3 \left(1 - \alpha \dk^{\frac 3 2 }\right)\le 0\\
% \alpha c_i(\xk) + M \dk^{\frac 1 2}\left(1 - \alpha \dk^{\frac 3 2 }\right)^2 + \epsilon_g \dk^{\frac 3 2} \left(1 - \alpha \dk^{\frac 3 2 }\right)\le 0\\
% \end{align*}
% 
% 
% \begin{align*}
% \alpha \dk^{\frac 3 2 } c_i(\xk) + M \frac {c_i(\xk)^2}{\|\gik\|^2}\left(1 - \alpha \dk^{\frac 3 2 }\right)^2 + \epsilon_g \dk^2 \left(1 - \alpha \dk^{\frac 3 2 }\right)\frac{-c_i(\xk)}{\|\gik\|} \le 0\\
% \alpha \dk^{\frac 3 2 }\|\gik\| + M \frac {c_i(\xk)}{\|\gik\|}\left(1 - \alpha \dk^{\frac 3 2 }\right)^2 - \epsilon_g \dk^2 \left(1 - \alpha \dk^{\frac 3 2 }\right) \ge 0\\
% M \dk\left(1 - \alpha \dk^{\frac 3 2 }\right)^2 + \epsilon_g \dk^2 \left(1 - \alpha \dk^{\frac 3 2 }\right) \le \alpha \dk^{\frac 3 2 }\|\gik\| \\
% \end{align*}


% \alpha \dk^{\frac 3 2 } c_i(\xk) + M \dk^2\left(1 - \alpha \dk^{\frac 3 2 }\right)^2 + \epsilon_g \dk^3 \left(1 - \alpha \dk^{\frac 3 2 }\right)\le 0\\
% \alpha c_i(\xk) + M \dk^{\frac 1 2}\left(1 - \alpha \dk^{\frac 3 2 }\right)^2 + \epsilon_g \dk^{\frac 3 2} \left(1 - \alpha \dk^{\frac 3 2 }\right)\le 0\\
% \epsilon_g \dk^2 \left(1 - \alpha \dk^{\frac 3 2 }\right) \le \alpha \dk^{\frac 3 2 }\|\gik\| + M \frac {c_i(\xk)}{\|\gik\|}\left(1 - \alpha \dk^{\frac 3 2 }\right)^2\\
% \epsilon_g \dk^2 \left(1 - \alpha \dk^{\frac 3 2 }\right) + M\sqrt{n} \dk \left(1 - \alpha \dk^{\frac 3 2 }\right)^2 \le \alpha \dk^{\frac 3 2 }\|\gik\|\\
% \epsilon_g \dk \left(1 - \alpha \dk^{\frac 3 2 }\right) + M\sqrt{n} \left(1 - \alpha \dk^{\frac 3 2 }\right)^2 \le \alpha \dk^{\frac 1 2 }\|\gik\|\\
% \epsilon_g \dk \left(1 - \alpha \dk^{\frac 3 2 }\right) + M\sqrt{n} \left(1 - 2\alpha \dk^{\frac 3 2 } + \alpha^2 \dk^{3}\right) \le \alpha \dk^{\frac 1 2 }\|\gik\|\\



% = \epsilon_g \dk^2 \nu^T\left(\wik - \xk\right) + c_i(\xk)\alpha \dk^{\frac 3 2} + M \left\|\wik - \xk\right\|^2 \\
% \le \epsilon_g \dk^3 + c_i(\xk)\alpha \dk^{\frac 3 2} + M \dk^2 \\
% = \dk^{\frac 3 2} \left(\epsilon_g \dk^{\frac 3 2} + c_i(\xk)\alpha + M \dk^{\frac 1 2}\right) \\


Also, by \cref{def_f} we know that $-s^T \hgik \ge \beta \dk^{\frac 1 2}$ where $\|s\| = 1$, and by \cref{def_g_bounds}, \cref{def_gk}, \cref{def_lambda_poised}, and our assumption \cref{delta_is_small_enough}, we know

% \Longrightarrow \dk^{\frac 3 2} \le \frac{\beta}{2\epsilon_{g}}g_{\text{low}}
% \Longrightarrow -s^T\gik \ge \frac 1 2 g_{\text{low}} \beta \dk^{\frac 1 2} + \epsilon_{g}\dk^2\nu^T s \\
\begin{align}
\dk \le \left(\frac{\beta}{2\epsilon_{g}}g_{\text{low}}\right)^{\frac 2 3} 
\Longrightarrow \dk^{\frac 3 2} \le \frac{\beta}{\epsilon_{g}}\left(g_{\text{low}} - \frac 1 2 g_{\text{low}}\right)
\Longrightarrow \frac{\epsilon_{g}}{\beta} \dk^{\frac 3 2} \le \|\gik\| - \frac 1 2 g_{\text{low}} \nonumber \\
\Longrightarrow -\|\gik\|s^T\hgik \ge \|\gik\|\beta\dk^{\frac 1 2} \ge \frac 1 2 g_{\text{low}} \beta \dk^{\frac 1 2} + \epsilon_{g}\dk^2 
\Longrightarrow -s^T\gik \ge \frac 1 2 g_{\text{low}} \beta \dk^{\frac 1 2} + \epsilon_{g}\dk^2|\nu^T s| \nonumber \\ 
\Longrightarrow -s^T\left(\gik + \epsilon_{g}\dk^2\nu\right) \ge \frac 1 2 g_{\text{low}} \beta \dk^{\frac 1 2} 
\Longrightarrow -s^T\nabla c_i(\xk) \ge \frac 1 2 g_{\text{low}} \beta \dk^{\frac 1 2}. \label{nsc_pos}
\end{align}

% So that if $\xk + ts \in B_{\infty}(\xk, \dk)$, then 

We also know by \cref{delta_is_small_enough} that
% \Longrightarrow \dk^{\frac 1 2} \le \frac {g_{\text{low}} \beta} {2M\sqrt{n}\left(1 + \frac {\lgi} M \right)}
\begin{align}
\dk \le \left[\frac {g_{\text{low}} \beta} {2M\sqrt{n}\left(1 + \frac {\lgi} M \right)} \right]^2 
\Longrightarrow \sqrt{n}\left(1 + \frac {\lgi} M \right) \dk^{\frac 1 2} \le \frac 1 {2M} g_{\text{low}} \beta \nonumber \\
\Longrightarrow \sqrt{n}\left(1 + \frac {\lgi} M \right) \dk \le \frac 1 {2M} g_{\text{low}} \beta \dk^{\frac 1 2}
\Longrightarrow \sqrt{n} \dk \le -\frac 1 M \sqrt{n}\dk \lgi + \frac 1 {2M} g_{\text{low}} \beta \dk^{\frac 1 2} \label{eqn2}
\end{align}
which implies by \cref{def_g_bounds}, \cref{t_is_bounded}, \cref{nsc_pos}, \cref{eqn2}

% t \le -\frac 1 M \left( \sqrt{n}\dk L_g + \nabla c_i(\xk)^Ts\right) \\
\begin{align}
t 
\le \sqrt{n} \dk 
\le -\frac 1 M \sqrt{n}\dk \lgi + \frac 1 {2M} g_{\text{low}} \beta \dk^{\frac 1 2} 
\le -\frac 1 M \sqrt{n}\dk \lgi -\frac 1 M \nabla c_i(\xk)^Ts \nonumber \\
\Longrightarrow \sqrt{n}\dk \lgi + \nabla c_i(\xk)^Ts + M t \le 0. \label{eqn3}
\end{align}

From there, we can conclude with \cref{def_lipshitz}, \cref{z_is_active}, \cref{eqn3}
\begin{align*}
c_i(y) = c_i(\wik + ts) \le c_i(\wik) + t\nabla c_i(\wik)^Ts + M t^2 \\
\le t \left(\nabla c_i(\wik)^Ts - \nabla c_i(\xk)^Ts + \nabla c_i(\xk)^Ts + M t\right) \\
\le t \left(\lgi\|\wik - \xk\| + \nabla c_i(\xk)^Ts + M t\right) \\
\le t \left(\sqrt{n}\dk \lgi + \nabla c_i(\xk)^Ts + M t\right) \le 0.
\end{align*}

% Thus, the entire cone $\fik$ is feasible with respect to the linearization of the constraint $c_i$ for sufficiently small $\dk$.
% \begin{align*}
% c_i(\xk + ts) \le c_i(\xk) + t\nabla c_i(\xk)^Ts + Mt^2 \le t(\nabla c_i(\xk)^Ts + Mt) \le 0.\\
% \end{align*}


\end{proof}



\begin{theorem}
We have that $\cap_{i \in \iik} \fik \cap B_{\infty}(\xk, \dk) \subseteq \f$ 
\end{theorem}

\begin{proof}
This follows directly from $x \in \fik \cap B_{\infty}(\xk, \dk) \Longrightarrow c_i(x) \le 0$.
\end{proof}



% 
% 
% 
% 
% 
% Thus, for sufficiently small $\dk$, we know that
% \begin{align*}
% \left(1 - \dk^{\frac 3 2 }\alpha\right)\frac{m_{c_i}(\xk)}{\|\gik\|} \le  \\
% \le \left(1 - \dk^{\frac 3 2 }\alpha\right)\dk \le \frac{4h_{\text{hi}}n}{g_{\text{low}}^4}\dk^2 \\
% \end{align*}
% \begin{align*}
% \left(1 - \dk^{\frac 3 2 }\alpha\right)\frac{m_{c_i}(\xk)}{\|\gik\|^2}\gik \ge \frac{4h_{\text{hi}}n}{g_{\text{low}}^4}\dk^2 \\
% \left(1 - \dk^{\frac 3 2 }\alpha\right)\frac{m_{c_i}(\xk)}{\|\gik\|} \ge \|\zik - z^{\star}\| \\
% \left(1 - \dk^{\frac 3 2 }\alpha\right)\sqrt{n}\dk \ge \|\zik - z^{\star}\| \\
% \end{align*}
% 
% \begin{align*}
% \|\zik - \xk\| \le \|\zik - z^{\star}\| + \|z^{\star}- \xk\|
% \end{align*}
% 
% But
% \begin{align*}
% \|z^{\star} - \xk\| = \|z^{\star} - \zik\| + \|\zik - \xk\| \\
% \|\zik - \xk\| \le \|z^{\star} - \xk\| \\
% \end{align*}
% 
% 
% \begin{align*}
% \|\xk + \left(1 - \dk^{\frac 3 2 }\alpha\right)\left(\zik - \xk\right) -\xk\| \le \|\xk - c_ -\xk\|\\
% \left| \right|
% \end{align*}
% 
% 
% \begin{align*}
% c_i(\xk) + \nabla c_i(\xk)^T(\xk - t \gik - \xk) = 0 \\
% c_i(\xk) - t \nabla c_i(\xk)^T\gik = 0 \\
% \frac{c_i(\xk)}{\nabla c_i(\xk)^T\gik} = t \\
% \end{align*}
% 
% |m_{c_i}(\xk)| \le \dk^2\left(\delta \|\gik\| + \epsilon_g\right) \\
% \frac{|m_{c_i}(\xk)|}{\|\gik\|} \le \dk^2\left(\delta + \frac{\epsilon_g}{\|\gik\|}\right) \\
% \sqrt{n}\dk \le \dk^2\left(\delta + \frac{\epsilon_g}{\|\gik\|}\right) \\
% \sqrt{n} \le \dk\left(\delta + \frac{\epsilon_g}{\|\gik\|}\right) \\
% \dk \ge\frac{\sqrt{n}}{\delta + \frac{\epsilon_g}{\|\gik\|}} \\

% \begin{align*}
% c_i(y) = c_i(\xk) + \nabla c_i(\xk)^T(y - \xk) + \frac 1 2 (y - \xk)^T\nabla^2c_i(\xi) (y - \xk) \\
% c_i(y) - \frac 1 2 (y - \xk)^T\nabla^2c_i(\xi) (y - \xk) = c_i(\xk) + \nabla c_i(\xk)^T(y - \xk) \\
% = c_i(\xk) + \nabla c_i(\xk)^T(y - z^i + z^i) \\
% = \nabla c_i(\xk)^T(y - \xk + \frac{c_i(\xk)}{\|\nabla c_i(\xk)\|^2}\nabla c_i(\xk) - \xk)\\
% = \nabla c_i(\xk)^T(\xk - (1 - \alpha\dk^{\frac 1 2})\frac {m_{c_i}(\xk)}{\|\nabla m_{c_i}(\xk)\|^2} \nabla m_{c_i}(\xk) - t\frac{\nabla m_{c_i}(\xk)}{\|\nabla m_{c_i}(\xk)\|} + s - \xk + \frac{c_i(\xk)}{\|\nabla c_i(\xk)\|^2}\nabla c_i(\xk) - \xk) \\
% = \nabla c_i(\xk)^T \left(\xk -\frac {m_{c_i}(\xk)}{\|\nabla m_{c_i}(\xk)\|^2} \nabla m_{c_i}(\xk)- \xk + \frac{c_i(\xk)}{\|\nabla c_i(\xk)\|^2}\nabla c_i(\xk) \right) \\
% + \nabla c_i(\xk)^T(\alpha\dk^{\frac 1 2}\frac {m_{c_i}(\xk)}{\|\nabla m_{c_i}(\xk)\|^2} \nabla m_{c_i}(\xk) - t\frac{\nabla m_{c_i}(\xk)}{\|\nabla m_{c_i}(\xk)\|} + s  - \xk)
% \end{align*}



% \le m_{c_i}(\xk) + \nabla c_i(\xk)^T(y - \xk) + M \|y - \xk\|^2 \\
% \le m_{c_i}(\xk) + \nabla c_i(\xk)^T((1 - \alpha\dk)\frac {m_{c_i}(\xk)}{\|\nabla m_{c_i}(\xk)\|^2} \nabla m_{c_i}(\xk) + t\frac{\nabla m_{c_i}(\xk)}{\|\nabla m_{c_i}(\xk)\|} + s) + M \|y - \xk\|^2

% \begin{align*}
% \dk \le \sqrt{\frac {1 + \delta} {2\epsilon_{g}}} \\
% 2\dk^2 - \beta \le \frac{1 + \delta}{\epsilon_{g}} \\
% \dk^2(2 - \beta\dk^{-2}) \le \frac{1 + \delta}{\epsilon_{g}} \\
% \epsilon_{g}\dk^2(t +  (1 - \beta\dk^{-2})t) \le t + \delta t \\
% \epsilon_{g}\dk^2(t + \|s\|) \le t + \delta t\\
% - \epsilon_{g}\dk^2\nu^T(t\frac{\nabla m_{c_i}(\xk)}{\|\nabla m_{c_i}(\xk)\|} + s) \le t + \delta t \\
% -t - \epsilon_{g}\dk^2\nu^T(t\frac{\nabla m_{c_i}(\xk)}{\|\nabla m_{c_i}(\xk)\|} + s) \le \delta t \\
% (\nabla m_{c_i}(\xk) + \epsilon_{g}\dk^2\nu)^T(- t\frac{\nabla m_{c_i}(\xk)}{\|\nabla m_{c_i}(\xk)\|} + s) \le \delta t \\
% \nabla c_i(\xk)^T(- t\frac{\nabla m_{c_i}(\xk)}{\|\nabla m_{c_i}(\xk)\|} + s) \le \delta t \\
% \end{align*}


\begin{theorem}
For sufficiently small $\dk$, the set $\fcki \subseteq \fik$ for all $1\le i \le m$.
\end{theorem}


\begin{proof}
Fix some $1\le i \le m$.
Letting $\dk \le 1$, we see that $\xk \in \fik$:
\begin{align*}
\xk = \xk + \left(1 - \alpha\dk^{\frac 1 2}\right)(\zik - \xk) - \left(1 - \alpha\dk^{\frac 1 2}\right)(\zik - \xk) \\
\end{align*}
where
\begin{align*}
\frac{-\left(1 - \alpha\dk^{\frac 1 2}\right)(\zik - \xk)}{\left\|-\left(1 - \alpha\dk^{\frac 1 2}\right)(\zik - \xk)\right\|}^T\hgik = 1 \ge \beta\\
\end{align*}

% = -\frac{m_{c_i}(\xk)}{\|\gik\|}\frac{\|\gik\|}{m_{c_i}(\xk)}

Now, let $\dk$ be sufficiently small that $-(\huk)^T\hgik \ge \beta\dk^{\frac 1 2}$.
This is possible, because $ -(\huk)^T\hgik \ge \theta^{\text{min}}_k \ge \theta^{\text{min}} > 0$.

Then
\begin{align*}
\left\{s\quad | \quad s^T(-\hgik)\ge\dk^{\frac 1 2}\beta\|s\| \right\}  \subseteq \left\{s\quad | \quad s^T\huk\ge\beta^{\star}\|s\| \right\}
\end{align*}
by \cref{cone_subset_cone} with $u^1 \gets \hat u$, $u^2 \gets -\hgik$, $\beta \gets \beta \dk^{\frac 1 2 }$ because:
\begin{align*}
\left(-\beta\dk^{\frac 1 2}(\huk)^T\hgik + \sqrt{(1 - \dk\beta^2)\left(1 - \left((\hgik)^T\hat u^{(k)}\right)^2\right)}\right) \\
\le \max_i \left(-\beta\dk^{\frac 1 2}(\huk)^T{\hgik} + \sqrt{(1 - \dk\beta^2)\left(1 - \left((\hgik)^T\huk\right)^2\right)}\right) \\
\le \beta\dk^{\frac 1 2} \max_i\{-(\huk)^T\hgik\} + \sqrt{(1 - \dk\beta^2)\left(1 - \left((\max_i(-\hgik)^T\huk\right)^2\right)} \\
= \beta\dk^{\frac 1 2} \theta^{\text{max}}_k + \sqrt{(1 - \dk\beta^2)\left(1 - (\theta^{\text{max}}_k) ^2\right)} = \beta^{\star}
\end{align*}
Thus, $\fcki \subseteq \fik$.
\end{proof}

\begin{theorem}
For sufficiently small $\dk$, $\fcki \ne \emptyset$.
\end{theorem}
\begin{proof}
As $\dk \to 0$, $\bsk \to \sqrt{1 - (\theta^{\text{max}}_k) ^2} \ge \bs > 0$.
% Also, the ray $\xk + t \huk$ is within this cone.
\end{proof}



\begin{theorem}
\label{ellipse_in_cone}
Let $r \le \delta$.
Then $\{x \in \rn | f_e(\delta, r, \theta; x) \le 0\} \subseteq \{tx\in\rn| e_1^T x \ge \theta,\|x\|=1, t>0\}$.
\end{theorem}


\begin{proof}
Let $x$ be such that $f_e(\delta, r, \theta, x) \le 0$.
First, note that
\begin{align*}
(e_1^Tx - \frac 1 2 \delta )^2\ge 0\\
\Longrightarrow (e_1^Tx)^2 + e_1^Tx\delta  - \frac 1 4 \delta^2 \ge 0\\
\Longrightarrow 0 \le 2(e_1^Tx)^2 + 2e_1^Tx\delta  - \frac 1 2 \delta^2\\
\Longrightarrow \frac 1 2 \delta^2 - (e_1^Tx)^2 + 2e_1^Tx\delta - \delta^2 \le (e_1^Tx)^2 \\
\Longrightarrow \frac 1 2 \delta^2 - \left((e_1^Tx)^2 - 2e_1^Tx\delta + \delta^2\right) \le (e_1^Tx)^2 \\
\Longrightarrow \frac 1 2 \delta^2 - (e_1^Tx - \delta)^2 \le (e_1^Tx)^2.
\end{align*}

We use this inequality to show:
\begin{align*}
f_e(\delta, r, \theta, x) \le 0 \\
\Longrightarrow (x - \delta e_1)^T\bigg(\begin{bmatrix}
1 & \boldsymbol0^T \\
\boldsymbol 0 & \frac{\theta^2}{1 - \theta^2} \boldsymbol I \\
\end{bmatrix}\bigg)(x - \delta e_1) \le \frac 1 2 r^2 \\
\Longrightarrow (e_1^Txe_1 + (x - e_1^Txe_1) - \delta e_1)^T\bigg(\begin{bmatrix}
1 & \boldsymbol0^T \\
\boldsymbol 0 & \frac{\theta^2}{1 - \theta^2} \boldsymbol I \\
\end{bmatrix}\bigg)(e_1^Txe_1 + (x - e_1^Txe_1) - \delta e_1) \le \frac 1 2 r^2 \\
\Longrightarrow
(e_1^Tx - \delta)^2 + \frac{\theta^2}{1 - \theta^2}\|x - e_1^Tx e_1\|^2 \le \frac 1 2 r^2 \\
\Longrightarrow
(e_1^Tx - \delta)^2 + \frac{\theta^2}{1 - \theta^2}(\|x\|^2 - (e_1^Tx)^2) \le \frac 1 2 \delta^2 \\
\Longrightarrow\frac{\theta^2}{1 - \theta^2}(\|x\|^2 - (e_1^Tx)^2) \le \frac 1 2 \delta^2 - (e_1^Tx - \delta)^2\\
\Longrightarrow\|x\|^2 - (e_1^Tx)^2 \le \frac{1 - \theta^2}{\theta^2}(e_1^Tx)^2 \\
\Longrightarrow\|x\|^2 \le \frac 1 {\theta^2}(e_1^Tx)^2 \\
\Longrightarrow e_1^T\frac{x}{\|x\|} \ge \theta.
\end{align*}
\end{proof}

\begin{theorem}
\label{ellipse_fits}
We have that $f_e(\delta, \sqrt{2}\delta, \theta; 0) = 0$ and $f_e(\delta, \delta, \theta; (1 + \frac{1}{\sqrt{2}}) \delta e_1) = 0$
\end{theorem}
\begin{proof}

We have
\begin{align*}
f_e(\delta, \sqrt{2}\delta, \theta; 0) =(0 - \delta e_1)^T\bigg(\begin{bmatrix}
1 & \boldsymbol0^T \\
\boldsymbol 0 & \frac{\theta^2}{1 - \theta^2} \boldsymbol I \\
\end{bmatrix}\bigg)(0 - \delta e_1) - \frac 1 2 (\sqrt 2 \delta)^2
=\delta^2 - \delta^2 = 0\\
\end{align*}
and
\begin{align*}
f_e(\delta, \delta, \theta; (1 + \frac{1}{\sqrt{2}}) \delta e_1) =\frac {\delta}{\sqrt{2}}e_1^T\bigg(\begin{bmatrix}
1 & \boldsymbol0^T \\
\boldsymbol 0 & \frac{\theta^2}{1 - \theta^2} \boldsymbol I \\
\end{bmatrix}\bigg)\frac {\delta}{\sqrt{2}}e_1 - \frac 1 2 \delta^2
=\frac 1 2 \delta^2 - \frac 1 2 \delta^2 = 0.\\
\end{align*}

\end{proof}

\begin{theorem}
For sufficiently small $\dk$, the ellipse $E_k \subseteq \f$ has a bounded condition number, and its major axis is atleast $\frac 1 2 \dk$
\end{theorem}

\begin{proof}
Note that the condition number is given by
\begin{align*}
\frac{\max\{1, \frac{\bs^2}{1 - \bs^2}\}}{\min\{1, \frac{\bs^2}{1 - \bs^2}\}}
\end{align*}
because the condition number of a matrix is not affected by rotations.

We have already shown that the cone $\fcki \cap B_{\infty}(\xk, \dk) \subset \f$.
By construction, the ellipse has been shifted to put $\xk$ at the origin and rotated by $R$ to put $\hat u$ along the x-axis.
Thus, by \cref{ellipse_in_cone}, we know that $E_k \subseteq \fcki$.
We know that $t_k > \dk$ because it measures the distance from $\xk$ to some point on the boundary of $B_{\infty}(\xk, \dk)$.
Also, we know from \cref{ellipse_fits} that the ellipse can be scaled by $2$ to include $\xk$, and extends by $\frac 1 2 \dk$ along $\huk$.
Finally, we know that it does not travel more than $\dk$ from $\xk$, so that it is contained within $B_{\infty}(\xk, \dk)$.
\end{proof}


\color{red}

% \begin{theorem}
% Suppose that there exists an $M$ such that $\|\xsk - \xk\| \le M$, and a $\delta > 0$ such that $\theta^{\text{min}}_k \ge \delta$.
% Then, there exists an $\epsilon > 0$, such that when $\dk \le \epsilon$, we have $ \fcko \cap B_{\infty}(\xk, \dk) \subseteq \mathcal F$.
% \end{theorem}
% 
% \begin{proof}
% Let
% \begin{align*}
% y = \xsk + \ask \hat u + ts \in \fcko \\
% s^{(i, k)} = \xsk + \ask\dk^{\frac 1 2}\huk - z^i + \alpha\dk^{\frac 1 2}\hgik\\
% \end{align*}
% 
% We can choose a sufficiently small $\dk$ so that
% \begin{align*}
% \dk \le (2\beta M)^{-\frac 2 3} \Longrightarrow \frac 1 2 \dk^{\frac 1 2 } \ge \dk^2 M\beta \\
% \dk \le \sqrt{\frac 1 {2\beta} \frac{\delta}{1 + \alpha + \delta\alpha}} \Longrightarrow \frac 1 2 \dk^{\frac 1 2 } \ge \beta\dk^{\frac 5 2}\frac{1 + \alpha + \delta\alpha}{\delta}.
% \end{align*}
% 
% Adding these together, we find that
% 
% \begin{align*}
% \dk^{\frac 1 2 } \ge \beta \dk^2 M + \beta \dk^{\frac 5 2}\frac{1 + \alpha + \delta\alpha}{\delta}.
% \end{align*}
% 
% But,
% \begin{align*}
% -(\hgik)^T s^{(i, k)} = -(\gik)^T \left(\xsk - \zik\right) + \dk^{\frac 1 2}\left(-(\hgik)^T\huk\ask - \alpha\right)
% \ge 0 + \dk^{\frac 1 2}\left(\theta^{\text{min}}_k \frac{1 + \alpha}{\theta^{\text{min}}_k} - \alpha\right) = \dk^{\frac 1 2 }\\
% \end{align*}
% and
% \begin{align*}
% \beta\dk^2\|s^{(i, k)}\| = \beta\dk^2\|\xsk + \ask\dk^{\frac 1 2}\huk - \zik + \alpha\dk^{\frac 1 2}\hgik \| \\
% \le \beta\dk^2\left(\|\xsk - \zik \| + \dk^{\frac 1 2} (\ask + \alpha)\right) 
% \le \dk^2\beta M + \beta \dk^{\frac 5 2}\frac{1 + \alpha + \delta\alpha}{\delta}
% \end{align*}
% imply 
% \begin{align*}
% \xsk + \ask\dk^{\frac 1 2}\huk \in \fik.
% \end{align*}
% 
% First, we note that
% \begin{align*}
% \left \{\xsk + \ask\dk^{\frac 1 2}\huk  + t s\quad | \quad \|s\| = 1, t> 0, -s^T(\hgik)\ge\dk^2\beta \right\} \\
% \subseteq \left \{\zik + \alpha\dk^{\frac 1 2}\huk + t s\quad | \quad \|s\| = 1, t> 0, -s^T(\hgik)\ge\dk^2\beta \right\}.  \\
% \end{align*}
% 
% In addition,
% \begin{align*}
% \left\{s\quad | \quad s^T(-\hgik)\ge\dk^2\beta\|s\| \right\}  \subseteq \left\{s\quad | \quad s^T\huk\ge\beta^{\star}\|s\| \right\}
% \end{align*}
% by \cref{cone_subset_cone}: (\color{red}we only need $-(\huk)^T\hgik \ge \beta$\color{black})
% \begin{align*}
% \left(-\beta\dk^2(\huk)^T\hgik + \sqrt{(1 - \dk^4\beta^2)\left(1 - \left((\hgik)^T\hat u^{(k)}\right)^2\right)}\right) \\
% \le \max_j \left(-\beta\dk^2(\huk)^T{\hgik} + \sqrt{(1 - \dk^4\beta^2)\left(1 - \left((\hgik)^T\huk\right)^2\right)}\right) \\
% \le \beta\dk^2 \max_j\{-(\huk)^T\hgik\} + \sqrt{(1 - \dk^4\beta^2)\left(1 - \left((\max_k(-\hgik)^T\huk\right)^2\right)} \\
% = \beta\dk^2 \theta^{\text{max}}_k + \sqrt{(1 - \dk^4\beta^2)\left(1 - (\theta^{\text{max}}_k) ^2\right)} = \beta^{\star}
% \end{align*}
% 
% Thus, $B(\xk, \dk) \cap \fcko \subseteq B(\xk, \dk) \cap \left(\cap_i \fik \right)\subseteq \f$.
% \end{proof}
% 
% 
% 
% 
% \color{red}
% 
% Because $\fcko$ is feasible, and $\xk \in \fcko$ we know that $ \fcki $ is feasible.
% But then, for any $\delta > 0$, the set is also feasible:
% \begin{align*}
% B_{\infty}(\xk, \dk) \cap \{x \in \rn | \quad f_e(\delta, r, \theta, R(x - \xk)) \le 0\}
% \end{align*}
% Choose $\delta$ so that this hits the edge of the trust region.
% 
% 
% \color{black}


\begin{theorem}
\end{theorem}

\begin{proof}
Given a point $c$, we wish to find the maximum-volume ellipsoid  $E \subset P$ centered at $c$, where the set $P$ is defined by $c_i$, $n_i$, $\beta_i$, $\|n_i\| = 1$:
\[
P = \{ x \in \rn \; | \;  x = c + c_i + ts_i, s_i^Tn_i \ge \beta_i, t>0, \|s_i\| = 1\quad \forall i = 1,\ldots,m\},
\]
we wish to find the maximum-volume ellipsoid $E \subset P$ centered at a point $c$.


Let $\bar{c_i} = c_i - c$ and $\bar x = x - c$ so that the set becomes
\[
P = \{ x \in \rn \; | \;  x = \bar c_i + ts_i, s_i^Tn_i \ge \beta_i, t>0, \|s_i\| = 1 \forall i = 1,\ldots,m\},
\]
The ellipsoid can then be centered at zero, and defined by a symmetric positive definite matrix $Q \succ 0$:
\[
E = \{ \bar x \; | \; \frac 1 2 \bar x^T Q \bar x \le 1 \}.
\]
Let 
\begin{align*}
g_i(\bar x) = \frac 1 2 \beta_i^2\|\bar x - \bar c_i\|^2 - \frac 1 2 \left((\bar x - \bar c_i)^Tn_i\right)^2 \\
g_i(\bar x) \le 0 
\Longrightarrow \beta_i^2\|\bar x - \bar c_i\|^2 \le \left((\bar x - \bar c_i)^Tn_i\right)^2 \Longrightarrow n_i^T\frac{\bar x - \bar c_i}{\|\bar x - \bar c_i\|}\ge \beta_i \\
\nabla g_i(\bar x) = \beta^2(\bar x - \bar c_i) - n_in_i^T(\bar x - \bar c_i) = (\beta^2 I - n_i n_i^T) \bar x + (n_in_i^T - I\beta^2) \bar c_i \\
= \left(\beta^2I - n_in_i^T\right)(\bar x - \bar c_i)
\end{align*}
so that the set $P = \{\bar x | g_i(\bar x) \le 0\}$.
Our goal is to determine $Q$ to maximize the volume of $E$ such that $\mu^{k} + E \subset P$.
Define the auxiliary function 
\[
f(\bar x) = \frac 1 2 \bar x^T Q \bar x
\]
so that 
\[
E = \{ \bar x \; | \; f(\bar x) \le 1 \}.
\]

\color{red}
Because $Q$ is positive definite, $f$ has a unique minimum on each cone 
\begin{align*}
C_i = \{ t, s | \bar c_i + tn_i + s_i, s_i^Tn_i = 0, t>0, \|s_i\| = \alpha_i t \}.
\end{align*}
\color{black}
Let this minimum be $\bar c_i + t_i s_i = d^{(i)} \in \argmin_{d \in C_i} f(d)$ for $i=1,\ldots,m$ where $\|s_i\| = 1$, $s_i^Tn_i \ge \beta$.
By the first order optimality conditions, there exists a $\lambda \in \mathbb R^m$ such that







\begin{align*}
\nabla f(d^{(i)}) = Q d^{(i)} = \lambda_i \left(\beta^2I - n_in_i^T\right)(d^{(i)} - \bar c_i) \\
\left[Q - \lambda_i (\beta^2 I - n_i n_i^T) \right]  d^{(i)} = \lambda_i \left(\beta^2I - n_in_i^T\right) \bar c_i \\
d^{(i)} = \left[Q - \lambda_i (\beta^2 I - n_i n_i^T) \right]^{-1}\lambda_i \left(\beta^2I - n_in_i^T\right) \bar c_i \\
\end{align*}


\begin{align*}
\nabla f(d^{(i)}) = Q d^{(i)} = \lambda_i \left(\beta^2I - n_in_i^T\right)(d^{(i)} - \bar c_i) \\
Q \bar c_i + Q s_i = \lambda_i \left(\beta^2I - n_in_i^T\right)s_i \\
\bar c_i + s_i = \lambda_i Q^{-1} \left(\beta^2I - n_in_i^T\right)s_i \\
\bar c_i =\left[ \lambda_i Q^{-1} \left(\beta^2I - n_in_i^T\right) - I \right]s_i \\
\bar c_i =\left[ \lambda_i \left(\beta^2Q^{-1} - Q^{-1}n_in_i^T\right) - I \right]s_i \\
s_i^T{\bar c_i} =\lambda_i \left(\beta^2s_i^TQ^{-1}s_i - s_i^TQ^{-1}n_in_i^Ts_i\right) + \|s_i\|^2 \\
\frac{s_i^T{\bar c_i} - \|s_i\|^2}{\beta^2s_i^TQ^{-1}s_i - s_i^TQ^{-1}n_in_i^Ts_i} =\lambda_i \\
\lambda_i = \frac{s_i^T{\bar c_i} - \|s_i\|^2}{\beta^2s_i^TQ^{-1}s_i - \beta s_i^TQ^{-1}n_i\|s_i\|} \\
\end{align*}


\begin{align*}
\bar c_i + s_i = \lambda_i Q^{-1} \left(\beta^2I - n_in_i^T\right)s_i \\
\bar c_i + s_i = \lambda_i \left(\beta^2Q^{-1}s_i - \beta Q^{-1}n_i\|s_i\|\right) \\
\end{align*}


\begin{align*}
\bar c_i + s_i =  \frac{s_i^T{\bar c_i} - \|s_i\|^2}{\beta^2s_i^TQ^{-1}s_i - \beta s_i^TQ^{-1}n_i\|s_i\|}  Q^{-1} \left(\beta^2I - n_in_i^T\right)s_i \\
\end{align*}


We also know that
\begin{align*}
n_i^T\frac{d^{(i)} - \bar c_i}{\|d^{(i)} - \bar c_i\|} = \beta_i \\
n_i^T\left((1 - \beta^2) \left[\frac 1 {\lambda_i} Q - \beta^2 I + n_i n_i^T \right]^{-1} \bar c_i\right) - n_i^T\bar c_i = \beta_i\left\|\left((1 - \beta^2) \left[\frac 1 {\lambda_i} Q - \beta^2 I + n_i n_i^T \right]^{-1} \bar c_i\right) - \bar c_i\right\| \\
n_i^T\left(\left[\frac 1 {\lambda_i} Q - \beta^2 I + n_i n_i^T \right]^{-1} \bar c_i\right) - \frac{n_i^T\bar c_i}{1-\beta^2} = \beta_i\left\|\left(\left[\frac 1 {\lambda_i} Q - \beta^2 I + n_i n_i^T \right]^{-1} \bar c_i\right) - \bar c_i\right\| \\
\end{align*}









\vspace{10cm}





\begin{align*}
\nabla f(d^{(i)}) = Q d^{(i)} = \lambda_i \left[ \beta^2(d^{(i)} - \bar c_i) - n_in_i^T(d^{(i)} - \bar c_i) \right]\quad \forall 1\le i\le m \\
Q \bar c_i + t_i Q s_i = t_i \lambda_i \left[ \beta^2s_i - n_in_i^Ts_i \right]\quad \forall 1\le i\le m \\
Q \bar c_i + t_i Q s_i = t_i \lambda_i\beta^2s_i - t_i \lambda_i\beta n_i \quad \forall 1\le i\le m \\
\left(t_i Q - t_i \lambda_i\beta^2I \right) s_i = - t_i \lambda_i\beta n_i - Q \bar c_i \quad \forall 1\le i\le m \\
t_i\left(\lambda_i\beta^2I - Q \right) s_i = t_i \lambda_i\beta n_i + Q \bar c_i \quad \forall 1\le i\le m \\
\left(\lambda_i\beta^2I - Q \right) s_i = \lambda_i\beta n_i + \frac 1 { t_i} Q \bar c_i \quad \forall 1\le i\le m \\
s_i = \left(\lambda_i\beta^2I - Q \right)^{-1}\left(\lambda_i\beta n_i + \frac 1 { t_i} Q \bar c_i\right) \quad \forall 1\le i\le m \\
s_i = \lambda_i\beta \left(\lambda_i\beta^2I - Q \right)^{-1}n_i + \frac 1 { t_i} \left(\lambda_i\beta^2I - Q \right)^{-1}Q \bar c_i \quad \forall 1\le i\le m \\
\end{align*}

Letting $\hat Q = \left(\lambda_i\beta^2I - Q \right)^{-1}$

Dotting with $n_i^T$
\begin{align*}
\beta = \lambda_i\beta n_i^T\hat Q n_i + \frac 1 { t_i} n_i^T\hat Q Q \bar c_i\quad \forall 1\le i\le m \\
\frac{\beta - \frac 1 { t_i} n_i^T\hat Q Q \bar c_i }{\beta n_i^T\hat Q n_i} = \lambda_i \quad \forall 1\le i\le m \\
\lambda_i = \frac{t_i \beta - \frac 1 { t_i} n_i^T\hat Q Q \bar c_i }{\beta n_i^T\hat Q n_i} \quad \forall 1\le i\le m \\
\end{align*}


Dotting with $s_i^T$
\begin{align*}
1 = \lambda_i\beta s_i^T\hat Q n_i + \frac 1 { t_i} s_i^T\hat Q Q \bar c_i\quad \forall 1\le i\le m \\
\end{align*}


Dotting with $s_i^T$:
\begin{align*}
\lambda_i\beta^2 - s_i^T Q s_i = \lambda_i\beta^2 + \frac 1 { t_i}s_i^T Q \bar c_i \quad \forall 1\le i\le m \\
t_i s_i^T Q s_i + s_i^T Q \bar c_i = 0 \quad \forall 1\le i\le m \\
{\bar s_i}^T Q d^{(i)} = 0 \quad \forall 1\le i\le m \\
t_i =- \frac{s_i^T Q \bar c_i}{s_i^T Q s_i} \quad \forall 1\le i\le m \\
\end{align*}

and $n_i^T$
\begin{align*}
\lambda_i\beta^3 - n_i^TQ s_i = \lambda_i\beta + \frac 1 { t_i} n_i^T Q \bar c_i \quad \forall 1\le i\le m \\
t_i\lambda_i\beta^3 = t_i\lambda_i\beta + n_i^T Q \bar c_i + t_in_i^TQ s_i\quad \forall 1\le i\le m \\
n_i^T Q \bar c_i + t_in_i^TQ s_i = t_i\lambda_i\beta^3 - t_i\lambda_i\beta \quad \forall 1\le i\le m \\
n_i^T Q d^{(i)} = \beta t_i\lambda_i(\beta^2 - 1) \quad \forall 1\le i\le m \\
\lambda_i = \frac{n_i^T Q d^{(i)}}{\beta t_i(\beta^2 - 1)} \quad \forall 1\le i\le m \\
\lambda_i = \frac{n_i^T Q \left(\bar c_i - \frac{s_i^T Q \bar c_i}{s_i^T Q s_i} s_i\right)}{\beta \left(- \frac{s_i^T Q \bar c_i}{s_i^T Q s_i}\right)(\beta^2 - 1)} \quad \forall 1\le i\le m \\
= \frac{n_i^T Q \left(\bar c_i s_i^T Q s_i - ({s_i^T Q \bar c_i}) s_i\right)}{\beta \left(-{s_i^T Q \bar c_i}\right)(\beta^2 - 1)} \\
= \frac{(n_i^T Q \bar c_i)(s_i^T Q s_i) - (n_i^T Q s_i) ({s_i^T Q \bar c_i})}{\beta \left({s_i^T Q \bar c_i}\right)(1 - \beta^2)} \\
\end{align*}
and ${\bar c_i}^T$:

\begin{align*}
\lambda_i\beta^2{\bar c_i}^Ts_i - {\bar c_i}^TQ s_i = \lambda_i\beta {\bar c_i}^T n_i + \frac 1 { t_i} {\bar c_i}^TQ \bar c_i \quad \forall 1\le i\le m \\
\lambda_i\beta\left[\beta{\bar c_i}^Ts_i - {\bar c_i}^T n_i \right]= {\bar c_i}^TQ s_i + \frac 1 { t_i} {\bar c_i}^TQ \bar c_i \quad \forall 1\le i\le m \\
\frac{(n_i^T Q \bar c_i)(s_i^T Q s_i) - (n_i^T Q s_i) ({s_i^T Q \bar c_i})}{\beta \left({s_i^T Q \bar c_i}\right)(1 - \beta^2)} \beta\left[\beta{\bar c_i}^Ts_i - {\bar c_i}^T n_i \right]= {\bar c_i}^TQ s_i + \frac 1 { t_i} {\bar c_i}^TQ \bar c_i \quad \forall 1\le i\le m \\
\frac{(n_i^T Q \bar c_i)(s_i^T Q s_i) - (n_i^T Q s_i) ({s_i^T Q \bar c_i})}{\left({s_i^T Q \bar c_i}\right)(1 - \beta^2)}\left[\beta{\bar c_i}^Ts_i - {\bar c_i}^T n_i \right]= {\bar c_i}^TQ s_i - \frac{s_i^T Q s_i}{s_i^T Q \bar c_i} {\bar c_i}^TQ \bar c_i \quad \forall 1\le i\le m \\
\left[(n_i^T Q \bar c_i)(s_i^T Q s_i) - (n_i^T Q s_i) ({s_i^T Q \bar c_i})\right]\left[\beta{\bar c_i}^Ts_i - {\bar c_i}^T n_i \right]= (1 - \beta^2)\left({\bar c_i}^TQ s_i\right)^2 - (1 - \beta^2)(s_i^T Q s_i){\bar c_i}^TQ \bar c_i \quad \forall 1\le i\le m \\
\end{align*}









\begin{align*}
{ d^{(i)}}^T Q d^{(i)} \ge 1 \\
\left(\bar c_i + t_i s_i \right)^T Q \left(\bar c_i + t_i s_i\right) \ge 1 \\
{\bar c_i}^TQ\bar c_i + 2 t_i {\bar c_i}^TQ s_i + t_i^2 s_i^TQs_i \ge 1 \\
{\bar c_i}^TQ\bar c_i + 2 \left(- \frac{s_i^T Q \bar c_i}{s_i^T Q s_i}\right) {\bar c_i}^TQ s_i + \left(- \frac{s_i^T Q \bar c_i}{s_i^T Q s_i}\right)^2 s_i^TQs_i \ge 1 \\
(s_i^T Q s_i){\bar c_i}^TQ\bar c_i - 2 \left({s_i^T Q \bar c_i}\right)^2  + \left({s_i^T Q \bar c_i}\right)^2  \ge s_i^T Q s_i \\
(s_i^T Q s_i){\bar c_i}^TQ\bar c_i - \left({s_i^T Q \bar c_i}\right)^2  \ge s_i^T Q s_i \\
(s_i^T Q s_i){\bar c_i}^TQ\bar c_i - \left({s_i^T Q \bar c_i}\right)^2  \ge s_i^T Q s_i \\
\end{align*}





and adding gives
\begin{align*}
\|r_i\|^2  + \alpha_i t_i = \lambda_i\left[r_i + \alpha n_i\right]^TQ^{-1}\left[\alpha_i n_i - r_i\right] \quad \forall 1\le i\le m \\
\|r_i\|^2  + \alpha_i t_i = \lambda_i\left[ \alpha_ir_i^T Q^{-1} n_i - r_i^T Q^{-1}r_i  + \alpha n_i^TQ^{-1}\alpha_i n_i - \alpha n_i^TQ^{-1}r_i   \right] \quad \forall 1\le i\le m \\
\|r_i\|^2  + \alpha_i t_i = \lambda_i\left(\alpha^2 n_i^TQ^{-1} n_i - r_i^T Q^{-1}r_i  \right) \quad \forall 1\le i\le m \\
t_i = \frac{-\alpha \pm \sqrt{\alpha_i^2 - 4\alpha_i^2()}}{2\alpha_i^2} \\
2\alpha_it_i = -1 \pm \sqrt{1 + 4\lambda(\alpha^2 n_i^TQ^{-1} n_i - r_i^T Q^{-1}r_i)} \\
\lambda_i = \frac{\|r_i\|^2  + \alpha_i t_i}{\alpha^2 n_i^TQ^{-1} n_i - r_i^T Q^{-1}r_i}
\end{align*}

Plugging in,
\begin{align*}
f_e(x) = (x - x^{\star} - \|\bar x - x^{\star} \|\hat u)^T\bigg(R^T\begin{bmatrix}
1 & \boldsymbol0^T \\
\boldsymbol 0 & \alpha^{-2} \boldsymbol I \\
\end{bmatrix}R\bigg)(x - x^{\star} - \|\bar x - x^{\star} \|\hat u) - \frac 1 2 r^2 \\
E = \{x \in \mathbb R^n | f(x) \le 0\} \\
t_in_i + r_i = \lambda_i \left[\alpha_i Q^{-1} n_i - Q^{-1}r_i \right] \quad \forall 1\le i\le m \\
t_in_i + r_i = (\|r_i\|^2  + \alpha_i t_i)\frac{\alpha_i Q^{-1} n_i - Q^{-1}r_i}{\alpha^2 n_i^TQ^{-1} n_i - r_i^T Q^{-1}r_i} \quad \forall 1\le i\le m \\
\frac{\bar x}{\|\bar x\|} = \frac{\alpha_i Q^{-1} n_i - Q^{-1}r_i}{\alpha^2 n_i^TQ^{-1} n_i - r_i^T Q^{-1}r_i} \quad \forall 1\le i\le m \\
\end{align*}

% \begin{align*}
% t_i = (\|r_i\|^2 + \alpha_i t_i)\frac{\alpha_i n_i^TQ^{-1} n_i - n_i^TQ^{-1}r_i}{\alpha^2 n_i^TQ^{-1} n_i - r_i^T Q^{-1}r_i} \quad \forall 1\le i\le m \\
% \|r\|^2 = (\|r_i\|^2  + \alpha_i t_i)\frac{\alpha_i r_i^TQ^{-1} n_i - r_i^TQ^{-1}r_i}{\alpha^2 n_i^TQ^{-1} n_i - r_i^T Q^{-1}r_i} \quad \forall 1\le i\le m \\
% t_i + \|r_i\|^2 = \|r_i\|^2  + \alpha_i t_i \quad \forall 1\le i\le m \\
% t_i = \|r_i\|^2  \quad \forall 1\le i\le m \\
% t_i = \frac 1 \alpha_i \quad \forall 1\le i\le m \\
% \end{align*}

% We also know that if we let $\bar Q^{-1} =  (n_in_i^T - I)Q^{-1}(n_in_i^T - I)^T$,
% \begin{align*}
% g_i(d^{(i)}) = 0 \\
% \|2 (n_in_i^T - I)d^{(i)}\|^2 = (\alpha_i n_i^Td^{(i)})^2 \\
% \|2 (n_in_i^T - I)\lambda_i Q^{-1}\left((4n_i^Tn_i - 2I)^T\|r_i\| + \alpha_i n_i\right)\|^2 = (\alpha_i n_i^T\lambda_i Q^{-1}\left((4n_i^Tn_i - 2I)^T\|r_i\| + \alpha_i n_i)\right)^2 \\
% \|2 (n_in_i^T - I)\lambda_i Q^{-1}\left((4n_i^Tn_i - 2I)^T\|r_i\| + \alpha_i n_i\right)\|^2 = (\alpha_i n_i^T\lambda_i Q^{-1}\left((4n_i^Tn_i - 2I)^T\|r_i\| + \alpha_i n_i)\right)^2 \\
% A_i^T \lambda_i Q^{-1}A_i = \bar{b_i} \\
% \lambda_i = \frac {\bar{b_i}}{A_i^T  Q^{-1}A_i} \\
% \end{align*}
so that 
\[
d^{(i)} = \lambda_i Q^{-1}A_i = \frac {\bar{b_i}}{A_i^T  Q^{-1}A_i}  Q^{-1}A_i \quad \forall 1\le i\le m.
\]

Because $E \subset P$, we also know that $f(\bar x) \ge 1$ for each $i$. Thus,
\begin{align*}
\frac 1 2 \bar x^{T} Q \bar x \ge 1 \\
\|\bar x \|\bar x^{T} Q \frac{\bar x}{\|\bar x\|} \ge 2 \\
\|\bar x\| \bar x^{T} \frac{\alpha_i n_i - r_i}{\alpha^2 n_i^TQ^{-1} n_i - r_i^T Q^{-1}r_i}\ge 2 \\
\|\bar x\| \bar x^{T} (\alpha_i n_i - r_i)\ge 2\left(\alpha^2 n_i^TQ^{-1} n_i - r_i^T Q^{-1}r_i \right)\\
\|\bar x\| (\alpha_i n_i + r_i) (\alpha_i n_i - r_i)\ge 2\left(\alpha^2 n_i^TQ^{-1} n_i - r_i^T Q^{-1}r_i\right) \\
\|\bar x\| (\alpha_i^2 - \|r_i\|^2)\ge 2\left(\alpha^2 n_i^TQ^{-1} n_i - r_i^T Q^{-1}r_i\right) \\
\alpha_i^2 \|\bar x\| (1 - t^2)\ge 2\left(\alpha^2 n_i^TQ^{-1} n_i - r_i^T Q^{-1}r_i\right) \\
\alpha_i^2 (1 - t^2)\ge 2\bar x(\alpha_i Q^{-1} n_i - Q^{-1}r_i) \\
\alpha_i^2 (1 - t^2)\ge 2(r + \alpha_i n_i)(\alpha_i Q^{-1} n_i - Q^{-1}r_i) \\
\frac 1 2 \alpha_i^2 (1 - t^2)\ge r_i^T(\alpha_i Q^{-1} n_i - Q^{-1}r_i) + \alpha_i n_i^T(\alpha_i Q^{-1} n_i - Q^{-1}r_i)  \\
\frac 1 2 \alpha_i^2 (1 - t^2)\ge \alpha_i^2 n_i^T Q^{-1} n_i - r_i^TQ^{-1}r_i \\
\frac 1 2 (1 - t^2) + \frac{1}{\alpha_i^2}r_i^TQ^{-1}r_i \ge n_i^T Q^{-1} n_i \\
\end{align*}
% \Longrightarrow \frac 1 2 \left(\alpha_i n_i + r_ i\right)^{T} Q (\alpha_i n_i + r_ i) \ge 1 \\
% \Longrightarrow \alpha_i \left(\alpha_i n_i + r_ i\right)^{T} Q n_i + \left(\alpha_i n_i + r_ i\right)^{T} Qr_ i \ge 2 \\
% \Longrightarrow \alpha_i ^2 n_i^{T} Q n_i + \alpha r_ i^{T} Q n_i + \alpha_i n_i^TQr_ i + r_ i^TQr_ i \ge 2 \\
% \Longrightarrow \alpha_i ^2 n_i^{T} Q n_i + 2\alpha r_ i^{T} Q n_i + r_ i^TQr_ i \ge 2 \\

\end{proof}







\subsection{}


We wish to find a $\uk$ such that $(-\gik)^T\uk$ is close to one.
% Let $\uk$ be a linear combination of $\gik$: $\uk = A^Tz$
\begin{align*}
\min_{\uk} \sum_i (1 + (\uk)^T\gik)^2 = \min_z (e + Az)^T(e + Az) = n + \min_z 2e^TAz + z^T A^TA z = f(z) \\
\nabla f(z) = 0 \\
\Longrightarrow 2 A^Te+2A^TAz = 0 \\
\Longrightarrow A^TAz = -A^Te \\
\Longrightarrow z = -\left(A^TA\right)^{-1} A^Te \\
\end{align*}

We need $-A(A^TA)^{-1}A^Te$ to be bounded.


\begin{align*}
-A(A^TA)^{-1}A^Te = 
-(U\Sigma V^T)((U\Sigma V)^T(U\Sigma V))^{-1}(U\Sigma V^T)^Te = 
-U\Sigma V^T(V\Sigma^2 V^T)^{-1}V\Sigma U^Te = \\
-U\Sigma V^TV\Sigma^{-2} V^TV\Sigma U^Te =
-U\Sigma \Sigma^{-2} \Sigma U^Te =
-U U^Te = -e
\end{align*}


\begin{align*}
-A^T(AA^T)^{-1}Ae = \\
-(U\Sigma V^T)^T((U\Sigma V^T)(U\Sigma V^T)^T)^{-1}(U\Sigma V^T)e = \\
-V\Sigma U^TU\Sigma^{-2} U^TU\Sigma V^Te = \\
-V\Sigma \Sigma^{-2} \Sigma V^Te = \\
-V V^Te = -e\\
\end{align*}





Let $u = -A^Tz$ be a linear combination of the negative gradients.
Then, 

\begin{align*}
\min_{\uk} \sum_i (1 - (-\gik)^Tu)^2 = \min_{\uk} \sum_i (1 - (\gik)^TA^Tz)^2 \\
\min_z (e + AA^Tz)^T(e + AA^Tz) = n + \min_z 2e^TAA^Tz + z^T AA^TAA^T z = f(z) \\
\nabla f(z) = 0 \\
\Longrightarrow 2 AA^Te+2AA^TAA^Tz = 0 \\
\Longrightarrow AA^TAA^Tz = -AA^Te \\
\end{align*}





% We know that for all $t \in \rn$:
% \begin{align*}
% c(\xk + t\gik) \le c_i(\xk) + \nabla c_i(\xk)^T\gik t + Mt^2. \\
% \end{align*}
% This quadratic has zeros at
% \begin{align*}
% \frac{1}{2M}\left(-\nabla c_i(\xk)^T\gik \pm \sqrt{\left(\nabla c_i(\xk)^T\gik\right)^2 - 4 M c_i(\xk)} \right).
% \end{align*}
% because $c_i(\xk) \le 0$.
% \color{red}For small enough $\dk$, we have that $\nabla c_i(\xk)^T\gik \ge 0 $, \color{black}so that
% \begin{align*}
% \sqrt{\left(\nabla c_i(\xk)^T\gik\right)^2 - 4 M c_i(\xk)} \ge -\nabla c_i(\xk)^T\gik \\
% \frac{1}{2M}\left(-\nabla c_i(\xk)^T\gik - \sqrt{\left(\nabla c_i(\xk)^T\gik\right)^2 - 4 M c_i(\xk)} \right) \le 0\\
% \end{align*}
% so that $c(\xk + t\gik) \le 0$ for all $t$ satisfying
% % Mt^2 - \nabla c_i(\xk)^T\hgik t + c_i(\xk) = 0 \\
% \begin{align*}
% 0 \le t \le \frac{1}{2M}\left(-\nabla c_i(\xk)^T\gik + \sqrt{\left(\nabla c_i(\xk)^T\gik\right)^2 - 4 M c_i(\xk)} \right). \\
% \end{align*}
% 
% For sufficiently small $\dk$, we know that
% 
% \begin{align*}
% \dk^2 \le \frac{1}{2\epsilon_g}\|\nabla c_i(\xk)\| \\
% \frac{1}{M}\epsilon_g \dk^2\|\nabla c_i(\xk)\| \le \frac{1}{2M}\|\nabla c_i(\xk)\|^2
% \end{align*}
% and
% \begin{align*}
% \dk + \frac{1}{2M}\epsilon_g \dk^2\|\nabla c_i(\xk)\|^2 \le \frac{1}{2M}\|\nabla c_i(\xk)\|^3 \\
% \dk \le \left(\|\nabla c_i(\xk)\| - \epsilon_g \dk^2\right) \frac{1}{2M}\|\nabla c_i(\xk)\|^2 \\
% \dk \le \left\|\nabla c_i(\xk) - \epsilon_g \dk^2\nu\right\| \frac{1}{2M}\|\nabla c_i(\xk)\|^2 \\
% \dk \frac{1}{\|\gik\|} \le \frac{1}{2M}\|\nabla c_i(\xk)\|^2.
% \end{align*}
% 
% Adding these, we find that
% \begin{align*}
% \dk \frac{1}{\|\gik\|} + \frac{1}{M}\epsilon_g \dk^2\|\nabla c_i(\xk)\| \le \frac{1}{M}\|\nabla c_i(\xk)\|^2 \\
% \dk \frac{1}{\|\gik\|} \le \frac{1}{M}\left(\|\nabla c_i(\xk)\|^2 - \epsilon_g \dk^2\|\nabla c_i(\xk)\|\right) \\
% \dk \frac{1}{\|\gik\|} \le \frac{1}{M}\left(\nabla c_i(\xk)^T(\nabla c_i(\xk) - \epsilon_g \dk^2 \nu)\right) \\
% \left(1 - \dk^{\frac 3 2 }\alpha\right)\frac{-m_{c_i}(\xk)}{\|\gik\|^2} \le \frac{1}{M}\left(\nabla c_i(\xk)^T\gik\right) \\
% \left(1 - \dk^{\frac 3 2 }\alpha\right)\frac{-m_{c_i}(\xk)}{\|\gik\|^2} \le \frac{1}{2M}\left(-\nabla c_i(\xk)^T\gik + \sqrt{\left(\nabla c_i(\xk)^T\gik\right)^2 - 4 M c_i(\xk)} \right) \\
% \end{align*}
% 
% 
% \begin{align*}
% \frac{4M^2n}{\|\gik\|}\dk^2 + 4M\sqrt{n} \nabla c_i(\xk)^T\gik \dk \le - 4 M c_i(\xk)\|\gik\| \\
% \frac{4M^2n}{\|\gik\|}\dk^2 + 4M\sqrt{n} \nabla c_i(\xk)^T\gik \dk \le - 4 M c_i(\xk)\|\gik\| \\
% \left(\frac{2M\sqrt{n}}{\|\gik\|}\dk\right)^2 + \frac{4M\sqrt{n}}{\|\gik\|}\dk \nabla c_i(\xk)^T\gik\le - 4 M c_i(\xk) \\
% % \left(\frac{2M\sqrt{n}}{\|\gik\|}\dk\right)^2 + 2 \frac{2M\sqrt{n}}{\|\gik\|}\dk \nabla c_i(\xk)^T\gik + \left(\nabla c_i(\xk)^T\gik\right)^2\le \left(\nabla c_i(\xk)^T\gik\right)^2 - 4 M c_i(\xk) \\
% \left(1 - \dk^{\frac 3 2 }\alpha\right)\frac{2M\sqrt{n}}{\|\gik\|}\dk +\nabla c_i(\xk)^T\gik \le \sqrt{\left(\nabla c_i(\xk)^T\gik\right)^2 - 4 M c_i(\xk)} \\
% % \left(1 - \dk^{\frac 3 2 }\alpha\right)\frac{-m_{c_i}(\xk)}{\|\gik\|^2} \le \frac{1}{2M}\left(-\nabla c_i(\xk)^T\gik + \sqrt{\left(\nabla c_i(\xk)^T\gik\right)^2 - 4 M c_i(\xk)} \right) \\
% \end{align*}


% We can conclude that
% \begin{align*}
% c_i\left(\xk + \left(1 - \alpha\dk^{\frac 3 2}\right)\left(\zik - \xk\right)\right) \le 0 \\
% c_i\left(\xk +\left(1 - \dk^{\frac 3 2 }\alpha\right)\frac{-m_{c_i}(\xk)}{\|\gik\|^2}\gik\right) \le 0 \\
% c_i\left(\xk + \left(1 - \alpha\dk^{\frac 3 2}\right)\left(\xk - \frac{m_{c_i}(\xk)}{\|\gik\|}\hgik - \xk\right)\right) \le 0 \\
% \end{align*}

% Redoing everything, we can just say
% \begin{align*}\
% c_i(\wik) = c_i\left(\xk + \left(1 - \alpha\dk^{\frac 3 2}\right)\left(\zik - \xk\right)\right) \\
%  = c_i\left(\xk + \left(1 - \alpha\dk^{\frac 3 2}\right)\left(\xk - \frac{m_{c_i}(\xk)}{\|\gik\|}\hgik - \xk\right)\right) \\
% = c_i\left(\xk + \left(1 - \dk^{\frac 3 2 }\alpha\right)\frac{-m_{c_i}(\xk)}{\|\gik\|^2}\gik\right) \\
% % \le c_i(\xk) + \left(1 - \dk^{\frac 3 2 }\alpha\right)\frac{-m_{c_i}(\xk)}{\|\gik\|^2}\nabla c_i(\xk)^T\gik + M\left[\left(1 - \dk^{\frac 3 2 }\alpha\right)\frac{-m_{c_i}(\xk)}{\|\gik\|}\right]^2 \\
% % \le c_i(\xk)\left[1 - \left(1 - \dk^{\frac 3 2 }\alpha\right)\frac{\nabla c_i(\xk)^T\gik}{\|\gik\|^2} + \frac{c_i(\xk)M}{\|\gik\|^2}\left(1 - \dk^{\frac 3 2 }\alpha\right)^2 \right]\\
% \end{align*}


% \begin{align*}
% 1 - \left(1 - \dk^{\frac 3 2 }\alpha\right)\frac{\nabla c_i(\xk)^T\gik}{\|\gik\|^2} + \frac{c_i(\xk)M}{\|\gik\|^2}\left(1 - \dk^{\frac 3 2 }\alpha\right)^2 \ge 0\\
% \|\gik\|^2 - \left(1 - \dk^{\frac 3 2 }\alpha\right)\nabla c_i(\xk)^T\gik + c_i(\xk)M\left(1 - \dk^{\frac 3 2 }\alpha\right)^2 \ge 0\\
% \left(1 - \dk^{\frac 3 2 }\alpha\right)\nabla c_i(\xk)^T\gik \le \|\gik\|^2 + c_i(\xk)M\left(1 - \dk^{\frac 3 2 }\alpha\right)^2\\
% \end{align*}


% 1 + \frac{c_i(\xk)M}{\|\gik\|^2}\left(1 - \dk^{\frac 3 2 }\alpha\right)^2 \ge \left(1 - \dk^{\frac 3 2 }\alpha\right)\frac{\nabla c_i(\xk)^T\gik}{\|\gik\|^2}\\
% \|\gik\|^2 + {c_i(\xk)M}\left(1 - \dk^{\frac 3 2 }\alpha\right)^2 \ge \left(1 - \dk^{\frac 3 2 }\alpha\right){\nabla c_i(\xk)^T\gik}\\
% \|\gik\|^2 + {c_i(\xk)M}\left(1 - \dk^{\frac 3 2 }\alpha\right)^2 \ge \left(1 - \dk^{\frac 3 2 }\alpha\right){\nabla c_i(\xk)^T\gik}\\

% \begin{align*}
% \nabla c_i(\xk)^T\gik + M\left[\left(1 - \dk^{\frac 3 2 }\alpha\right)\frac{-m_{c_i}(\xk)}{\|\gik\|^2}\right] \le 0 \\
% % \left(1 - \dk^{\frac 3 2 }\alpha\right)\frac{-m_{c_i}(\xk)}{\|\gik\|^2}\nabla c_i(\xk)^T\gik + M\left[\left(1 - \dk^{\frac 3 2 }\alpha\right)\frac{-m_{c_i}(\xk)}{\|\gik\|^2}\right]^2 \le 0 \\
% \end{align*}



% \Longrightarrow -\alpha \|\gik\| + M \left(1 - \alpha \dk^{\frac 3 2 }\right)^2 \sqrt{n}\dk^{\frac 1 2 }+ \epsilon_g \dk^{\frac 3 2} \left(1 - \alpha \dk^{\frac 3 2 }\right) & \le & 0 \\




% \begin{align*}
% c_i(\xk) + \nabla c_i(\xk)^T\left(\wik - \xk\right) = c_i(\xk) + \left(\gik +  \nu \epsilon_g \dk^2 \right)^T\left(\wik - \xk\right) \\
% = c_i(\xk) + \left(\gik\right)^T\left(\wik - \xk\right) + \epsilon_g \dk^2\nu^T \left(\wik - \xk\right) \\
% = \epsilon_g \dk^2 \nu^T\left(\wik - \xk\right) + c_i(\xk)\alpha \dk^{\frac 3 2} \\
% \end{align*}

% so that

