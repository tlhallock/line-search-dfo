
\newcommand{\gik}{{g^{(i, k)}}}
\newcommand{\hgik}{{{\hat g}^{(i, k)}}}
\newcommand{\zik}{{z^{(i, k)}}}
\newcommand{\fik}{{\mathcal F_{i, k}}}
\newcommand{\iik}{{\mathcal I_{k}}}
\newcommand{\uk}{{u^{(k)}}}
\newcommand{\huk}{{{\hat u}^{(k)}}}
\newcommand{\ask}{{\alpha^{(\star, k)}}}
\newcommand{\bsk}{{\beta^{(\star, k)}}}
\newcommand{\bs}{{\beta^{\star}}}
\newcommand{\fcko}{{\mathcal {F}^{\text{outer}}_k}}
\newcommand{\fcki}{{\mathcal {F}^{\text{inner}}_k}}
\newcommand{\rn}{{\mathbb R^{n}}}
\newcommand{\xsk}{{x^{(\star, k)}}}
\newcommand{\bxk}{{\bar{x}^{(k)}}}
\newcommand{\f}{{\mathcal F}}

\subsection{Notation}

Fix some $\alpha \in (0, 1)$, and $\beta \in (0, 1)$.
Then for all $1 \le i \le m$, $k = 1, \ldots$, define the following whenever they exist:
\begin{align*}
\f = \{x \in \rn | \quad c_i(x) \le 0\quad \forall 1 \le i \le m \} \\
\gik = \nabla m_{c_i}(\xk) \\ 
\hgik = \frac{\gik}{\|\gik\|} \\
\zik = \xk - \frac{m_{c_i}(\xk)}{\|\gik\|}\hgik \\
\fik = \left\{x \in \rn | x = \xk + \left(1 - \alpha\dk^{\frac 3 2}\right)\left(\zik - \xk\right) + t s,t > 0, \|s\| = 1, -s^T\hgik \ge \beta \dk^{\frac 3 2 }\right\} \\
\iik = \{1\le i \le m | \zik \in B_{\infty}(\xk, \dk) \} \\
\uk = -\nabla \hat m_{c_{\mathcal I}}(\xk)^T\left(\nabla \hat m_{c_{\mathcal I}}(\xk)\nabla \hat m_{c_{\mathcal I}}(\xk)^T\right)^{-1} e\\
\huk = \frac {\uk} {\| \uk\|}\\
\theta^{\text{min}}_k = \min_{i \in \mathcal I} (-\hgik)^T \huk \\
\theta^{\text{max}}_k = \max_{i \in \mathcal I} (-\hgik)^T \huk \\
\theta^{\text{min}} = \liminf_{k\to\infty} \theta^{\text{min}}_k\\
\theta^{\text{max}} = \limsup_{k\to\infty} \theta^{\text{max}}_k\\
\bs = \sqrt{ 1 - (\theta^{\text{max}}_k)^2}\\
\bsk = \beta\dk^{\frac 3 2} \theta^{\text{max}}_k + \sqrt{(1 - \dk^3\beta^2)\left(1 - (\theta^{\text{max}}_k) ^2\right)} \\
\fcki = \{x \in \rn | \quad x = \xk + ts, t > 0, \|s\| = 1, s^T\huk \ge \bsk \} \\
g_{\text{low}} \le \|\gik\| \le g_{\text{hi}} \\
h_{\text{low}} \le \|\nabla^2m_{c_i}(x)\| \le h_{\text{hi}} \\
f_e(\delta, r, \theta; x) = (x - \delta e_1)^T\bigg(\begin{bmatrix}
1 & \boldsymbol0^T \\
\boldsymbol 0 & \frac{\theta^2}{1 - \theta^2} \boldsymbol I \\
\end{bmatrix}\bigg)(x - \delta e_1) - \frac 1 2 r^2 \\
R = 2\frac{(e_1 + \huk)(e_1 + \huk)^T}{(e_1 + \huk)^T(e_1 + \huk)} - \boldsymbol I \\
t_k = \max_{\xk + t\huk \in B_{\infty}(\xk, \dk)} t \\
E_k = \{x \in \rn | f_e\left(\frac 1 2 t_k, \frac 1 2 t_k, \bsk; R(x - \xk)\right) \le 0\}
\end{align*}


% \fcko = \{x \in \rn | \quad x = \xsk + \ask\dk^{\frac 1 2}\huk + ts, t > 0, \|s\| = 1, s^T\huk \ge \bsk \} \\
% \ask = \frac{1 + \alpha}{\theta^{\text{min}}_k} \\
% \bxk = \xsk + (\huk)^T(\xk - \xsk)\huk\\
% m_{\mathcal I}(\xk) + \nabla m_{\mathcal I}(\xk)^T (\xsk - \xk) = 0 \\

\subsection{Derivation of feasible ellipse}


\begin{theorem}
We have $\theta^{\text{min}} > 0$ and $\theta^{\text{max}} < 1$.
\end{theorem}

\begin{proof}
Because the constraints are uniformly linearly independent, we know that there is an $\epsilon > 0$ such that for all $1 \le i, j \le m$, $i\ne j$ we have 
\begin{align*}
1 - (\hgik)^T \hat g^{(j, k)} \ge \epsilon \\
\end{align*}
\color{red}
This (somehow) means that 
\begin{align*}
\left(\nabla \hat m_{c_{\mathcal I}}(\xk)\nabla \hat m_{c_{\mathcal I}}(\xk)^T\right)^{-1} e \ge 0
\end{align*}
and $\huk$ is written as a linear combination of the negative gradients.
\begin{align*}
1 - (\huk)^T (-\hat g^{(j, k)}) \ge \epsilon_2 \\
\Longrightarrow (\huk)^T \hat g^{(j, k)} \le 1 - \epsilon
\end{align*}
Need:
\begin{align*}
(-\hgik)^T \huk \ge \epsilon\\
\theta^{\text{min}}_k \ge \epsilon \\
\end{align*}
\color{black}
\end{proof}




\begin{theorem}
\label{cone_subset_cone}
Given $u^1, u^2 \in \rn$, $\|u^1\| = \|u^2\|= 1$, $\beta >0$, with ${u^1}^Tu^2 \ge \beta$ define
\begin{align*}
B = \{x\in\rn | {u^2}^Tx \ge \beta\|x\|\}, \quad
S = \left\{x\in\rn \bigg| {u^1}^Tx \ge \left(\beta {u^1}^Tu^2 + \sqrt{(1 - \beta^2)\left(1 - ({u^2}^Tu^1)^2\right)}\right)\|x\| \right\}. 
\end{align*}
We have $S \subseteq B$.
\end{theorem}


\begin{proof}
For a contradiction, let $y \in \rn$ be such that $y \not \in B$ and $y \in S$.
Then $\hat y = \frac{y}{\|y\|} \in U$ where:
\begin{align*}
U = \left\{x \in \rn | {u^2}^Tx \le \beta, {u^1}^Tx \ge \beta {u^1}^Tu^2 + \sqrt{(1 - \beta^2)\left(1 - ({u^2}^Tu^1)^2\right)} \right\}.
\end{align*}

Note that
\begin{align*}
x^{\star} = \beta u^2 + \sqrt{\frac{1 - \beta^2}{1 - ({u^2}^Tu^1)^2}} (u^1 - {u^2}^Tu^1 u^2 )
\end{align*}
is on the boundary of of $U$ because

\begin{align*}
{u^1}^T\left(\beta u^2 + \sqrt{\frac{1 - \beta^2}{1 - ({u^2}^Tu^1)^2}} (u^1 - {u^2}^Tu^1 u^2 )\right) = 
\beta {u^1}^Tu^2 + \sqrt{(1 - \beta^2)\left(1 - ({u^2}^Tu^1)^2\right)} \\
{u^2}^T\left(\beta u^2 + \sqrt{\frac{1 - \beta^2}{1 - ({u^2}^Tu^1)^2}} (u^1 - {u^2}^Tu^1 u^2 )\right) = 
\beta + \sqrt{\frac{1 - \beta^2}{1 - ({u^2}^Tu^1)^2}} ({u^2}^Tu^1 - {u^2}^Tu^1 ) = \beta.
\end{align*}
$U$ is the intersection of two half-spaces, so $\hat y$ can be written in the form $\hat y = x^{\star} + v$ where ${u^1}^Tv \ge 0, {u^2}^Tv < 0$.
But since $v \ne 0$, we also know
\begin{align*}
{v}^Tx^{\star} = 
\left(\beta {v}^Tu^2 + \sqrt{\frac{1 - \beta^2}{1 - ({u^2}^Tu^1)^2}} ({v}^Tu^1 - {u^2}^Tu^1 {v}^Tu^2 )\right) > 0
\end{align*}
as ${u^1}^Tu^2 \ge \beta \Longrightarrow {u^2}^Tu^1\sqrt{\frac{1 - \beta^2}{1 - ({u^2}^Tu^1)^2}} \ge \beta$.
This implies that $1 = \|\hat y\| = \|x^{\star} + v\| > \|x^{\star}\| = 1$.
\end{proof}






\begin{theorem}
There exists an $\epsilon > 0$, such that if $\dk < \epsilon$ and $z^{(i, k)} \in B_{\infty}(\xk, \dk))$,
then $\fik \cap B_{\infty}(\xk, \dk) \subseteq \f$.

\end{theorem}

\begin{proof}
Let $y = z^i - \alpha \dk^{\frac 3 2}\hgik + t s \in \fik \cap B_{\infty}(\xk, \dk) $ with $t > 0, \|s\| = 1, -s^T\hgik \ge \beta \dk^2$, and let $M \ge \sup_{x \in B_{\infty}(\xk, \dk)} \frac 1 2 \nabla^2 c_i(x)$.
There exists a $\nu \in \rn$, $\|\nu\|=1$ such that 
\begin{align*}
\nabla c_i(\xk) = \nabla m_{c_i}(\xk) + \epsilon_{g}\dk^2\nu.
\end{align*}
We will let $z^{\star} = \xk + \frac{c_i(\xk)}{\nabla c_i(\xk) ^T \gik}\gik$.
Observe that
\begin{align*}
c_i(\xk) + \nabla c_i(\xk)^T (z^{\star} - \xk) = c_i(\xk) + \nabla c_i(\xk)^T\left( \xk - \frac{c_i(\xk)}{\nabla c_i(\xk) ^T \gik}\gik - \xk\right) \\
= c_i(\xk) - \nabla c_i(\xk)^T\left(\frac{c_i(\xk)}{\nabla c_i(\xk) ^T \gik}\gik\right) = c_i(\xk) - \frac{c_i(\xk)}{\nabla c_i(\xk) ^T \gik}\nabla c_i(\xk)^T\gik = 0. \\
\end{align*}
We also know that,
\begin{align*}
c_i(\xk) + \nabla c_i(\xk)^T (\xk - 0 \gik -\xk) \le 0
\end{align*}
because $\xk$ is feasible.
Thus, in order to show that 
\begin{align*}
c_i(\xk) + \nabla c_i(\xk)^T\left(\xk + \left(1 - \dk^{\frac 3 2 }\alpha\right)\left(\zik - \xk\right) - \xk\right) \le 0 \\
c_i(\xk) + \nabla c_i(\xk)^T\left(\xk + \left(1 - \dk^{\frac 3 2 }\alpha\right)\left(\xk - \frac{m_{c_i}(\xk)}{\|\gik\|}\hgik - \xk\right) - \xk\right) \le 0 \\
c_i(\xk) + \nabla c_i(\xk)^T\left(\xk - \left(1 - \dk^{\frac 3 2 }\alpha\right)\frac{m_{c_i}(\xk)}{\|\gik\|^2}\gik - \xk\right) \le 0
\end{align*}
we require
\begin{align*}
\left(1 - \dk^{\frac 3 2 }\alpha\right)\frac{-m_{c_i}(\xk)}{\|\gik\|^2} \le \frac{-c_i(\xk)}{\nabla c_i(\xk) ^T \gik} \\
\left(1 - \dk^{\frac 3 2 }\alpha\right)\nabla c_i(\xk) ^T \gik \le \|\gik\|^2  \\
\dk^{\frac 3 2 }\alpha \nabla c_i(\xk) ^T \gik \ge \nabla c_i(\xk) ^T \gik - \|\gik\|^2 \\
\dk^{\frac 3 2 }\alpha \left(\gik + \epsilon_{g}\dk^2\nu\right) ^T \gik \ge (\gik)^T\left(\nabla c_i(\xk) -\gik\right) \\
\dk^{\frac 3 2 }\alpha \left(\|\gik\|^2 + \epsilon_{g}\dk^2\nu^T\gik\right)  \ge \epsilon_{g}\dk^2(\gik)^T\nu \\
\alpha\|\gik\|^2 + \alpha\epsilon_{g}\dk^2\nu^T\gik  \ge \epsilon_{g}\dk^{\frac 1 2}(\gik)^T\nu \\
\alpha\|\gik\|^2 \ge \epsilon_{g}(\gik)^T\nu\left(\dk^{\frac 1 2} - \alpha\dk^2\right)  \\
\alpha\|\gik\| \ge \epsilon_{g}(\hgik)^T\nu\left(\dk^{\frac 1 2} - \alpha\dk^2\right)  \\
\end{align*}
This is true whenever $\dk$ is sufficiently small that
\begin{align*}
\dk^{\frac 1 2} - \alpha\dk^2 \le \frac{\alpha\|g_{\text{low}} \|}{\epsilon_{g}} \le \frac{\alpha\|\gik\|}{\epsilon_{g}}. \\
\end{align*}


We have just shown that, 
\begin{align*}
\xk + \left(1 - \dk^{\frac 3 2 }\alpha\right)\left(\zik - \xk\right)
\end{align*}
is feasible with respect to the linearized constraints for sufficiently small $\dk$.

Also, if $-s^T \hgik \ge \beta \dk^{\frac 3 2}$ where $\|s\| = 1$, then for sufficiently small $\dk$:
\begin{align*}
\dk \le \frac{\beta^2\|g_{\text{low}}\|^2}{\epsilon_g^2} 
\Longrightarrow \dk^{\frac 1 2 } \le \frac{\beta\|\gik\|}{\epsilon_g}
\Longrightarrow \beta \ge \frac{\epsilon_{g}\dk^{\frac 1 2}}{\|\gik\|}
\Longrightarrow \beta \ge \frac{\epsilon_{g}\dk^{\frac 1 2}}{\|\gik\|}
\Longrightarrow \beta\dk^{\frac 3 2} \ge \frac{\epsilon_{g}\dk^2}{\|\gik\|} \\
\Longrightarrow -s^T\hgik \ge \frac{\epsilon_{g}\dk^2}{\|\gik\|} 
\Longrightarrow -s^T\gik \ge \epsilon_{g}\dk^2|\nu^T s|
\Longrightarrow -s^T\gik \ge -\epsilon_{g}\dk^2\nu^T s \\
\Longrightarrow -s^T\gik + \epsilon_{g}\dk^2\nu^T s \ge 0
\Longrightarrow -\nabla c_i(\xk)^T s \ge 0.
\end{align*}

Thus, the entire cone $\fik$ is feasible with respect to the linearization of the constraint $c_i$ for sufficiently small $\dk$.




\color{red}



% \dk^2 \le \frac{\|\gik\| -\delta e } {2\epsilon_{g} + \beta \|\gik\|} \\
% \left(2\epsilon_{g} + \beta \|g^i\|\right)\dk^2 - \|\gik\|\le -\delta e \\
% 2\epsilon_{g}\dk^2 + \|\gik\|(\beta \dk^2 - 1)\le -\delta e \\
% (\epsilon_{g}\dk^2\nu)^T (s - \hgik) + \|\gik\|({\hat g}^i)^T(s - \hgik)\le -\delta e \\
% (\gik + \epsilon_{g}\dk^2\nu)^T (s - \hgik) \le -\delta e \\
% -\nabla c_i(\xk)^T s \ge -\delta e


Alternatively,
\begin{align*}
\dk \le \sqrt{\frac{\sqrt{n}}{\epsilon_gg_{\text{hi}}}} \\
\dk^2 \le \frac{\sqrt{n}}{\epsilon_g\|\gik\|} \\
\epsilon_g\dk^2 \le \frac{2\sqrt{n}}{\|\gik\|} - \frac{\sqrt{n}}{\|\gik\|} \\
\sqrt{n}\frac{1}{\|\gik\|} \le \frac{2\sqrt{n}}{\|\gik\|}  - \epsilon_g\dk^2 \\
\sqrt{n}\frac{\dk}{\|\gik\|} \le \frac{2\sqrt{n}}{\|\gik\|^2}\dk \|\gik\| - \epsilon_g\dk^3 \\
|m_{c_i}(\xk)| \le \frac{2\sqrt{n}}{\|\gik\|^2} \|\gik\| - \epsilon_g\dk^3 \\
|m_{c_i}(\xk)|\|\gik\| \le \frac{2\sqrt{n}}{\|\gik\|^2} \|\gik\|^2 + \epsilon_g\dk^3\nu^T\gik \\
|m_{c_i}(\xk)|\|\gik\| \le \frac{2\sqrt{n}}{\|\gik\|^2} \dk(\gik + \epsilon_g\nu\dk^2)^T \gik\\
|m_{c_i}(\xk)|\|\gik\| \le \frac{2\sqrt{n}}{\|\gik\|^2}\dk \nabla c_i(\xk) ^T \gik\\
\|\xk - \frac{c_i(\xk)}{\nabla c_i(\xk) ^T \gik}\gik- \xk\| \le \frac{2\sqrt{n}}{\|\gik\|^2}\dk  \le \frac{2\sqrt{n}}{g_{\text{low}}^2}\dk \\
\|z^{\star} - \xk\| \le \frac{2\sqrt{n}}{\|\gik\|^2}\dk  \le \frac{2\sqrt{n}}{g_{\text{low}}^2}\dk \\
\end{align*}

We also know that
\begin{align*}
\left\|\xk - \frac{m_{c_i}(\xk)}{\|\gik\|}\hgik - \xk + \frac{c_i(\xk)}{\nabla c_i(\xk) ^T \gik}\gik\right\| \le h_{\text{hi}}\left\|\xk - \frac{c_i(\xk)}{\nabla c_i(\xk) ^T \gik}\gik- \xk\right\|^2
\le \frac{4h_{\text{hi}}n}{g_{\text{low}}^4}\dk^2 \\
\|z^{\star} - \zik\| \le \|z^{\star} - \xk\|^2 \le \frac{4h_{\text{hi}}n}{g_{\text{low}}^4}\dk^2 \\
\end{align*}


However, 
\begin{align*}
\left\|\xk + \left(1 - \dk^{\frac 3 2 }\alpha\right)(\zik - \xk) - \zik\right\| \\
=\left\|\left(1 - \dk^{\frac 3 2 }\alpha\right)(\zik - \xk) - (\zik - \xk)\right\| \\
=\|\xk - \zik\|\left|\left(1 - \dk^{\frac 3 2 }\alpha\right) - 1\right| \\
=\dk^{\frac 3 2 }\alpha\|\xk - \zik\| \\
\end{align*}
and
\begin{align*}
\left\|\xk + \left(1 -  \dk^{\frac 3 2 }\alpha\right)(\zik - \xk) - \xk\right\| = \left(1 - \alpha\dk^{\frac 3 2 }\right)\|\xk - \zik\|.\\
\end{align*}


% \le \sqrt{n} \dk^{\frac 3 2 }\alpha\dk \\


\color{black}



% 
% 
% 
% 
% 
% Thus, for sufficiently small $\dk$, we know that
% \begin{align*}
% \left(1 - \dk^{\frac 3 2 }\alpha\right)\frac{m_{c_i}(\xk)}{\|\gik\|} \le  \\
% \le \left(1 - \dk^{\frac 3 2 }\alpha\right)\dk \le \frac{4h_{\text{hi}}n}{g_{\text{low}}^4}\dk^2 \\
% \end{align*}
% \begin{align*}
% \left(1 - \dk^{\frac 3 2 }\alpha\right)\frac{m_{c_i}(\xk)}{\|\gik\|^2}\gik \ge \frac{4h_{\text{hi}}n}{g_{\text{low}}^4}\dk^2 \\
% \left(1 - \dk^{\frac 3 2 }\alpha\right)\frac{m_{c_i}(\xk)}{\|\gik\|} \ge \|\zik - z^{\star}\| \\
% \left(1 - \dk^{\frac 3 2 }\alpha\right)\sqrt{n}\dk \ge \|\zik - z^{\star}\| \\
% \end{align*}
% 
% \begin{align*}
% \|\zik - \xk\| \le \|\zik - z^{\star}\| + \|z^{\star}- \xk\|
% \end{align*}
% 
% But
% \begin{align*}
% \|z^{\star} - \xk\| = \|z^{\star} - \zik\| + \|\zik - \xk\| \\
% \|\zik - \xk\| \le \|z^{\star} - \xk\| \\
% \end{align*}
% 
% 
% \begin{align*}
% \|\xk + \left(1 - \dk^{\frac 3 2 }\alpha\right)\left(\zik - \xk\right) -\xk\| \le \|\xk - c_ -\xk\|\\
% \left| \right|
% \end{align*}
% 
% 
% \begin{align*}
% c_i(\xk) + \nabla c_i(\xk)^T(\xk - t \gik - \xk) = 0 \\
% c_i(\xk) - t \nabla c_i(\xk)^T\gik = 0 \\
% \frac{c_i(\xk)}{\nabla c_i(\xk)^T\gik} = t \\
% \end{align*}
% 
% |m_{c_i}(\xk)| \le \dk^2\left(\delta \|\gik\| + \epsilon_g\right) \\
% \frac{|m_{c_i}(\xk)|}{\|\gik\|} \le \dk^2\left(\delta + \frac{\epsilon_g}{\|\gik\|}\right) \\
% \sqrt{n}\dk \le \dk^2\left(\delta + \frac{\epsilon_g}{\|\gik\|}\right) \\
% \sqrt{n} \le \dk\left(\delta + \frac{\epsilon_g}{\|\gik\|}\right) \\
% \dk \ge\frac{\sqrt{n}}{\delta + \frac{\epsilon_g}{\|\gik\|}} \\

% \begin{align*}
% c_i(y) = c_i(\xk) + \nabla c_i(\xk)^T(y - \xk) + \frac 1 2 (y - \xk)^T\nabla^2c_i(\xi) (y - \xk) \\
% c_i(y) - \frac 1 2 (y - \xk)^T\nabla^2c_i(\xi) (y - \xk) = c_i(\xk) + \nabla c_i(\xk)^T(y - \xk) \\
% = c_i(\xk) + \nabla c_i(\xk)^T(y - z^i + z^i) \\
% = \nabla c_i(\xk)^T(y - \xk + \frac{c_i(\xk)}{\|\nabla c_i(\xk)\|^2}\nabla c_i(\xk) - \xk)\\
% = \nabla c_i(\xk)^T(\xk - (1 - \alpha\dk^{\frac 1 2})\frac {m_{c_i}(\xk)}{\|\nabla m_{c_i}(\xk)\|^2} \nabla m_{c_i}(\xk) - t\frac{\nabla m_{c_i}(\xk)}{\|\nabla m_{c_i}(\xk)\|} + s - \xk + \frac{c_i(\xk)}{\|\nabla c_i(\xk)\|^2}\nabla c_i(\xk) - \xk) \\
% = \nabla c_i(\xk)^T \left(\xk -\frac {m_{c_i}(\xk)}{\|\nabla m_{c_i}(\xk)\|^2} \nabla m_{c_i}(\xk)- \xk + \frac{c_i(\xk)}{\|\nabla c_i(\xk)\|^2}\nabla c_i(\xk) \right) \\
% + \nabla c_i(\xk)^T(\alpha\dk^{\frac 1 2}\frac {m_{c_i}(\xk)}{\|\nabla m_{c_i}(\xk)\|^2} \nabla m_{c_i}(\xk) - t\frac{\nabla m_{c_i}(\xk)}{\|\nabla m_{c_i}(\xk)\|} + s  - \xk)
% \end{align*}



% \le m_{c_i}(\xk) + \nabla c_i(\xk)^T(y - \xk) + M \|y - \xk\|^2 \\
% \le m_{c_i}(\xk) + \nabla c_i(\xk)^T((1 - \alpha\dk)\frac {m_{c_i}(\xk)}{\|\nabla m_{c_i}(\xk)\|^2} \nabla m_{c_i}(\xk) + t\frac{\nabla m_{c_i}(\xk)}{\|\nabla m_{c_i}(\xk)\|} + s) + M \|y - \xk\|^2

% \begin{align*}
% \dk \le \sqrt{\frac {1 + \delta} {2\epsilon_{g}}} \\
% 2\dk^2 - \beta \le \frac{1 + \delta}{\epsilon_{g}} \\
% \dk^2(2 - \beta\dk^{-2}) \le \frac{1 + \delta}{\epsilon_{g}} \\
% \epsilon_{g}\dk^2(t +  (1 - \beta\dk^{-2})t) \le t + \delta t \\
% \epsilon_{g}\dk^2(t + \|s\|) \le t + \delta t\\
% - \epsilon_{g}\dk^2\nu^T(t\frac{\nabla m_{c_i}(\xk)}{\|\nabla m_{c_i}(\xk)\|} + s) \le t + \delta t \\
% -t - \epsilon_{g}\dk^2\nu^T(t\frac{\nabla m_{c_i}(\xk)}{\|\nabla m_{c_i}(\xk)\|} + s) \le \delta t \\
% (\nabla m_{c_i}(\xk) + \epsilon_{g}\dk^2\nu)^T(- t\frac{\nabla m_{c_i}(\xk)}{\|\nabla m_{c_i}(\xk)\|} + s) \le \delta t \\
% \nabla c_i(\xk)^T(- t\frac{\nabla m_{c_i}(\xk)}{\|\nabla m_{c_i}(\xk)\|} + s) \le \delta t \\
% \end{align*}

\end{proof}

\begin{theorem}
For sufficiently small $\dk$, the set $\fcki \subseteq \fik$ for all $1\le i \le m$.
\end{theorem}


\begin{proof}
Fix some $1\le i \le m$.
Letting $\dk \le 1$, we see that $\xk \in \fik$:
\begin{align*}
\xk = \xk + \left(1 - \alpha\dk^{\frac 3 2}\right)(\zik - \xk) - \left(1 - \alpha\dk^{\frac 3 2}\right)(\zik - \xk) \\
\end{align*}
where
\begin{align*}
\frac{\left(1 - \alpha\dk^{\frac 3 2}\right)(\zik - \xk)}{\left\|\left(1 - \alpha\dk^{\frac 3 2}\right)(\zik - \xk)\right\|}^T\hgik =\frac{-m_{c_i}(\xk)}{\|\gik\|}\frac{1}{\|\zik - \xk\|} = 1 \ge \beta\\
\end{align*}

% = -\frac{m_{c_i}(\xk)}{\|\gik\|}\frac{\|\gik\|}{m_{c_i}(\xk)}

Now, let $\dk$ be sufficiently small that $-(\huk)^T\hgik \ge \beta\dk^{\frac 3 2}$.
This is possible, because $ -(\huk)^T\hgik \ge \theta^{\text{min}}_k \ge \theta^{\text{min}} > 0$.

Then
\begin{align*}
\left\{s\quad | \quad s^T(-\hgik)\ge\dk^{\frac 3 2}\beta\|s\| \right\}  \subseteq \left\{s\quad | \quad s^T\huk\ge\beta^{\star}\|s\| \right\}
\end{align*}
by \cref{cone_subset_cone} with $u^1 \gets \hat u$, $u^2 \gets -\hgik$, $\beta \gets \beta \dk^{\frac 3 2 }$ because:
\begin{align*}
\left(-\beta\dk^{\frac 3 2}(\huk)^T\hgik + \sqrt{(1 - \dk^3\beta^2)\left(1 - \left((\hgik)^T\hat u^{(k)}\right)^2\right)}\right) \\
\le \max_i \left(-\beta\dk^{\frac 3 2}(\huk)^T{\hgik} + \sqrt{(1 - \dk^3\beta^2)\left(1 - \left((\hgik)^T\huk\right)^2\right)}\right) \\
\le \beta\dk^{\frac 3 2} \max_i\{-(\huk)^T\hgik\} + \sqrt{(1 - \dk^3\beta^2)\left(1 - \left((\max_i(-\hgik)^T\huk\right)^2\right)} \\
= \beta\dk^{\frac 3 2} \theta^{\text{max}}_k + \sqrt{(1 - \dk^3\beta^2)\left(1 - (\theta^{\text{max}}_k) ^2\right)} = \beta^{\star}
\end{align*}
Thus, $\fcki \subseteq \fik$.
\end{proof}

\begin{theorem}
For sufficiently small $\dk$, $\fcki \ne \emptyset$, $\fcki \cap B_{\infty}(\xk, \dk) \subseteq \f$, and the cone contains one point on the boundary of $B_{\infty}(\xk, \dk)$.
\end{theorem}
\begin{proof}
As $\dk \to 0$, $\bsk \to \sqrt{1 - (\theta^{\text{max}}_k) ^2} \ge \bs > 0$. Also, the ray $\xk + t \huk$ is within this cone.
\end{proof}



\begin{theorem}
\label{ellipse_in_cone}
Let $r \le \delta$.
Then $\{x \in \rn | f_e(\delta, r, \theta; x) \le 0\} \subseteq \{tx\in\rn| e_1^T x \ge \theta,\|x\|=1, t>0\}$.
\end{theorem}


\begin{proof}
Let $x$ be such that $f_e(\delta, r, \theta, x) \le 0$.
First, note that
\begin{align*}
\Longrightarrow (e_1^Tx - \frac 1 2 \delta )^2\ge 0\\
\Longrightarrow (e_1^Tx)^2 + e_1^Tx\delta  - \frac 1 4 \delta^2 \ge 0\\
\Longrightarrow 0 \le 2(e_1^Tx)^2 + 2e_1^Tx\delta  - \frac 1 2 \delta^2\\
\Longrightarrow \frac 1 2 \delta^2 - (e_1^Tx)^2 + 2e_1^Tx\delta - \delta^2 \le (e_1^Tx)^2 \\
\Longrightarrow \frac 1 2 \delta^2 - \left((e_1^Tx)^2 - 2e_1^Tx\delta + \delta^2\right) \le (e_1^Tx)^2 \\
\Longrightarrow \frac 1 2 \delta^2 - (e_1^Tx - \delta)^2 \le (e_1^Tx)^2.
\end{align*}

We use this inequality to show:
\begin{align*}
f_e(\delta, r, \theta, x) \le 0 \\
\Longrightarrow (x - \delta e_1)^T\bigg(\begin{bmatrix}
1 & \boldsymbol0^T \\
\boldsymbol 0 & \frac{\theta^2}{1 - \theta^2} \boldsymbol I \\
\end{bmatrix}\bigg)(x - \delta e_1) \le \frac 1 2 r^2 \\
\Longrightarrow (e_1^Txe_1 + (x - e_1^Txe_1) - \delta e_1)^T\bigg(\begin{bmatrix}
1 & \boldsymbol0^T \\
\boldsymbol 0 & \frac{\theta^2}{1 - \theta^2} \boldsymbol I \\
\end{bmatrix}\bigg)(e_1^Txe_1 + (x - e_1^Txe_1) - \delta e_1) \le \frac 1 2 r^2 \\
\Longrightarrow
(e_1^Tx - \delta)^2 + \frac{\theta^2}{1 - \theta^2}\|x - e_1^Tx e_1\|^2 \le \frac 1 2 r^2 \\
\Longrightarrow
(e_1^Tx - \delta)^2 + \frac{\theta^2}{1 - \theta^2}(\|x\|^2 - (e_1^Tx)^2) \le \frac 1 2 \delta^2 \\
\Longrightarrow\frac{\theta^2}{1 - \theta^2}(\|x\|^2 - (e_1^Tx)^2) \le \frac 1 2 \delta^2 - (e_1^Tx - \delta)^2\\
\Longrightarrow\|x\|^2 - (e_1^Tx)^2 \le \frac{1 - \theta^2}{\theta^2}(e_1^Tx)^2 \\
\Longrightarrow\|x\|^2 \le \frac 1 {\theta^2}(e_1^Tx)^2 \\
\Longrightarrow e_1^T\frac{x}{\|x\|} \ge \theta
\end{align*}
\end{proof}

\begin{theorem}
\label{ellipse_fits}
We have that $f_e(\delta, \sqrt{2}\delta, \theta; 0) = 0$ and $f_e(\delta, \delta, \theta; (1 + \frac{1}{\sqrt{2}}) \delta e_1) = 0$
\end{theorem}
\begin{proof}

We have
\begin{align*}
f_e(\delta, \sqrt{2}\delta, \theta; 0) =(0 - \delta e_1)^T\bigg(\begin{bmatrix}
1 & \boldsymbol0^T \\
\boldsymbol 0 & \frac{\theta^2}{1 - \theta^2} \boldsymbol I \\
\end{bmatrix}\bigg)(0 - \delta e_1) - \frac 1 2 (\sqrt 2 \delta)^2
=\delta^2 - \delta^2 = 0\\
\end{align*}
and
\begin{align*}
f_e(\delta, \delta, \theta; (1 + \frac{1}{\sqrt{2}}) \delta e_1) =\frac {\delta}{\sqrt{2}}e_1^T\bigg(\begin{bmatrix}
1 & \boldsymbol0^T \\
\boldsymbol 0 & \frac{\theta^2}{1 - \theta^2} \boldsymbol I \\
\end{bmatrix}\bigg)\frac {\delta}{\sqrt{2}}e_1 - \frac 1 2 \delta^2
=\frac 1 2 \delta^2 - \frac 1 2 \delta^2 = 0.\\
\end{align*}

\end{proof}

\begin{theorem}
For sufficiently small $\dk$, the ellipse $E_k \subseteq \f$ has a bounded condition number, and its major axis is atleast $\frac 1 2 \dk$
\end{theorem}

\begin{proof}
Note that the condition number is given by
\begin{align*}
\frac{\max\{1, \frac{\bs^2}{1 - \bs^2}\}}{\min\{1, \frac{\bs^2}{1 - \bs^2}\}}
\end{align*}
because the condition number of a matrix is not affected by rotations.

We have already shown that the cone $\fcki \cap B_{\infty}(\xk, \dk) \subset \f$.
By construction, the ellipse has been shifted to put $\xk$ at the origin and rotated by $R$ to put $\hat u$ along the x-axis.
Thus, by \cref{ellipse_in_cone}, we know that $E_k \subseteq \fcki$.
We know that $t_k > \dk$ because it measures the distance from $\xk$ to some point on the boundary of $B_{\infty}(\xk, \dk)$.
Also, we know from \cref{ellipse_fits} that the ellipse can be scaled by $2$ to include $\xk$, and extends by $\frac 1 2 t_k$ along $\huk$.
Finally, we know that it does not travel more than $\frac 1 2 t_k$, \color{red} which is well away from the boundary of $B_{\infty}(\xk, \dk)$ \color{black}.
\end{proof}


\color{red}

\begin{theorem}
Suppose that there exists an $M$ such that $\|\xsk - \xk\| \le M$, and a $\delta > 0$ such that $\theta^{\text{min}}_k \ge \delta$.
Then, there exists an $\epsilon > 0$, such that when $\dk \le \epsilon$, we have $ \fcko \cap B_{\infty}(\xk, \dk) \subseteq \mathcal F$.
\end{theorem}

\begin{proof}
Let
\begin{align*}
y = \xsk + \ask \hat u + ts \in \fcko \\
s^{(i, k)} = \xsk + \ask\dk^{\frac 1 2}\huk - z^i + \alpha\dk^{\frac 1 2}\hgik\\
\end{align*}

We can choose a sufficiently small $\dk$ so that
\begin{align*}
\dk \le (2\beta M)^{-\frac 2 3} \Longrightarrow \frac 1 2 \dk^{\frac 1 2 } \ge \dk^2 M\beta \\
\dk \le \sqrt{\frac 1 {2\beta} \frac{\delta}{1 + \alpha + \delta\alpha}} \Longrightarrow \frac 1 2 \dk^{\frac 1 2 } \ge \beta\dk^{\frac 5 2}\frac{1 + \alpha + \delta\alpha}{\delta}.
\end{align*}

Adding these together, we find that

\begin{align*}
\dk^{\frac 1 2 } \ge \beta \dk^2 M + \beta \dk^{\frac 5 2}\frac{1 + \alpha + \delta\alpha}{\delta}.
\end{align*}

But,
\begin{align*}
-(\hgik)^T s^{(i, k)} = -(\gik)^T \left(\xsk - \zik\right) + \dk^{\frac 1 2}\left(-(\hgik)^T\huk\ask - \alpha\right)
\ge 0 + \dk^{\frac 1 2}\left(\theta^{\text{min}}_k \frac{1 + \alpha}{\theta^{\text{min}}_k} - \alpha\right) = \dk^{\frac 1 2 }\\
\end{align*}
and
\begin{align*}
\beta\dk^2\|s^{(i, k)}\| = \beta\dk^2\|\xsk + \ask\dk^{\frac 1 2}\huk - \zik + \alpha\dk^{\frac 1 2}\hgik \| \\
\le \beta\dk^2\left(\|\xsk - \zik \| + \dk^{\frac 1 2} (\ask + \alpha)\right) 
\le \dk^2\beta M + \beta \dk^{\frac 5 2}\frac{1 + \alpha + \delta\alpha}{\delta}
\end{align*}
imply 
\begin{align*}
\xsk + \ask\dk^{\frac 1 2}\huk \in \fik.
\end{align*}

First, we note that
\begin{align*}
\left \{\xsk + \ask\dk^{\frac 1 2}\huk  + t s\quad | \quad \|s\| = 1, t> 0, -s^T(\hgik)\ge\dk^2\beta \right\} \\
\subseteq \left \{\zik + \alpha\dk^{\frac 1 2}\huk + t s\quad | \quad \|s\| = 1, t> 0, -s^T(\hgik)\ge\dk^2\beta \right\}.  \\
\end{align*}

In addition,
\begin{align*}
\left\{s\quad | \quad s^T(-\hgik)\ge\dk^2\beta\|s\| \right\}  \subseteq \left\{s\quad | \quad s^T\huk\ge\beta^{\star}\|s\| \right\}
\end{align*}
by \cref{cone_subset_cone}: (\color{red}we only need $-(\huk)^T\hgik \ge \beta$\color{black})
\begin{align*}
\left(-\beta\dk^2(\huk)^T\hgik + \sqrt{(1 - \dk^4\beta^2)\left(1 - \left((\hgik)^T\hat u^{(k)}\right)^2\right)}\right) \\
\le \max_j \left(-\beta\dk^2(\huk)^T{\hgik} + \sqrt{(1 - \dk^4\beta^2)\left(1 - \left((\hgik)^T\huk\right)^2\right)}\right) \\
\le \beta\dk^2 \max_j\{-(\huk)^T\hgik\} + \sqrt{(1 - \dk^4\beta^2)\left(1 - \left((\max_k(-\hgik)^T\huk\right)^2\right)} \\
= \beta\dk^2 \theta^{\text{max}}_k + \sqrt{(1 - \dk^4\beta^2)\left(1 - (\theta^{\text{max}}_k) ^2\right)} = \beta^{\star}
\end{align*}

Thus, $B(\xk, \dk) \cap \fcko \subseteq B(\xk, \dk) \cap \left(\cap_i \fik \right)\subseteq \f$.
\end{proof}




\color{red}

Because $\fcko$ is feasible, and $\xk \in \fcko$ we know that $ \fcki $ is feasible.
But then, for any $\delta > 0$, the set is also feasible:
\begin{align*}
B_{\infty}(\xk, \dk) \cap \{x \in \rn | \quad f_e(\delta, r, \theta, R(x - \xk)) \le 0\}
\end{align*}
Choose $\delta$ so that this hits the edge of the trust region.





\begin{theorem}
\end{theorem}

\begin{proof}
Given a point $c$, we wish to find the maximum-volume ellipsoid  $E \subset P$ centered at $c$, where the set $P$ is defined by $c_i$, $n_i$, $\beta_i$, $\|n_i\| = 1$:
\[
P = \{ x \in \rn \; | \;  x = c + c_i + ts_i, s_i^Tn_i \ge \beta_i, t>0, \|s_i\| = 1\quad \forall i = 1,\ldots,m\},
\]
we wish to find the maximum-volume ellipsoid $E \subset P$ centered at a point $c$.


Let $\bar{c_i} = c_i - c$ and $\bar x = x - c$ so that the set becomes
\[
P = \{ x \in \rn \; | \;  x = \bar c_i + ts_i, s_i^Tn_i \ge \beta_i, t>0, \|s_i\| = 1 \forall i = 1,\ldots,m\},
\]
The ellipsoid can then be centered at zero, and defined by a symmetric positive definite matrix $Q \succ 0$:
\[
E = \{ \bar x \; | \; \frac 1 2 \bar x^T Q \bar x \le 1 \}.
\]
Let 
\begin{align*}
g_i(\bar x) = \frac 1 2 \beta_i^2\|\bar x - \bar c_i\|^2 - \frac 1 2 \left((\bar x - \bar c_i)^Tn_i\right)^2 \\
g_i(\bar x) \le 0 
\Longrightarrow \beta_i^2\|\bar x - \bar c_i\|^2 \le \left((\bar x - \bar c_i)^Tn_i\right)^2 \Longrightarrow n_i^T\frac{\bar x - \bar c_i}{\|\bar x - \bar c_i\|}\ge \beta_i \\
\nabla g_i(\bar x) = \beta(\bar x - \bar c_i) - n_in_i^T(\bar x - \bar c_i) = (\beta I - n_i n_i^T) \bar x + (1 - \beta) \bar c_i
\end{align*}
so that the set $P = \{\bar x | g_i(\bar x) \le 0\}$.
Our goal is to determine $Q$ to maximize the volume of $E$ such that $\mu^{k} + E \subset P$.
Define the auxiliary function 
\[
f(\bar x) = \frac 1 2 \bar x^T Q \bar x
\]
so that 
\[
E = \{ \bar x \; | \; f(\bar x) \le 1 \}.
\]

\color{red}
Because $Q$ is positive definite, $f$ has a unique minimum on each cone 
\begin{align*}
C_i = \{ t, s | \bar c_i + tn_i + s_i, s_i^Tn_i = 0, t>0, \|s_i\| = \alpha_i t \}.
\end{align*}
\color{black}
Let this minimum be $t_i s_i = d^{(i)} \in \argmin_{d \in C_i} f(d)$ for $i=1,\ldots,m$ where $\|s_i\| = 1$, $s_i^Tn_i \ge \beta$.
By the first order optimality conditions, there exists a $\lambda \in \mathbb R^m$ such that
\begin{align*}
\nabla f(d^{(i)}) = Q d^{(i)} = \lambda_i \left[(\beta I - n_i n_i^T) d^{(i)} + (1 - \beta) \bar c_i \right]\quad \forall 1\le i\le m \\
\end{align*}
Dotting with $r_i$
\begin{align*}
r_i^T\left[I + \lambda_i Q^{-1}\right]r_i = \lambda_i\alpha_i r_i^TQ^{-1} n_i \quad \forall 1\le i\le m \\
\|r_i\|^2 = - \lambda_ir_i Q^{-1}r_i + \lambda_i\alpha_i r_i^TQ^{-1} n_i \quad \forall 1\le i\le m \\
\|r_i\|^2 = - \lambda_ir_i Q^{-1}\left[r_i + \alpha_i n_i\right] \quad \forall 1\le i\le m \\
\|r_i\|^2 = \lambda_ir_i Q^{-1}\left[\alpha_i n_i - r_i\right] \quad \forall 1\le i\le m
\end{align*}

and $n_i$
\begin{align*}
n_i^Tt_in_i + \left[n_i^T + \lambda_i n_i^TQ^{-1}\right]r_i = \lambda_i\alpha_i n_i^TQ^{-1} n_i \quad \forall 1\le i\le m \\
t_i + \lambda_i n_i^TQ^{-1}r_i = \lambda_i\alpha_i n_i^TQ^{-1} n_i \quad \forall 1\le i\le m \\
t_i = \lambda_i\alpha_i n_i^TQ^{-1} n_i - \lambda_i n_i^TQ^{-1}r_i \quad \forall 1\le i\le m \\
t_i = \lambda_in_i^TQ^{-1}\left[\alpha_i n_i - r_i\right] \quad \forall 1\le i\le m \\
\end{align*}

and adding gives
\begin{align*}
\|r_i\|^2  + \alpha_i t_i = \lambda_i\left[r_i + \alpha n_i\right]^TQ^{-1}\left[\alpha_i n_i - r_i\right] \quad \forall 1\le i\le m \\
\|r_i\|^2  + \alpha_i t_i = \lambda_i\left[ \alpha_ir_i^T Q^{-1} n_i - r_i^T Q^{-1}r_i  + \alpha n_i^TQ^{-1}\alpha_i n_i - \alpha n_i^TQ^{-1}r_i   \right] \quad \forall 1\le i\le m \\
\|r_i\|^2  + \alpha_i t_i = \lambda_i\left(\alpha^2 n_i^TQ^{-1} n_i - r_i^T Q^{-1}r_i  \right) \quad \forall 1\le i\le m \\
t_i = \frac{-\alpha \pm \sqrt{\alpha_i^2 - 4\alpha_i^2()}}{2\alpha_i^2} \\
2\alpha_it_i = -1 \pm \sqrt{1 + 4\lambda(\alpha^2 n_i^TQ^{-1} n_i - r_i^T Q^{-1}r_i)} \\
\lambda_i = \frac{\|r_i\|^2  + \alpha_i t_i}{\alpha^2 n_i^TQ^{-1} n_i - r_i^T Q^{-1}r_i}
\end{align*}

Plugging in,
\begin{align*}
f_e(x) = (x - x^{\star} - \|\bar x - x^{\star} \|\hat u)^T\bigg(R^T\begin{bmatrix}
1 & \boldsymbol0^T \\
\boldsymbol 0 & \alpha^{-2} \boldsymbol I \\
\end{bmatrix}R\bigg)(x - x^{\star} - \|\bar x - x^{\star} \|\hat u) - \frac 1 2 r^2 \\
E = \{x \in \mathbb R^n | f(x) \le 0\} \\
t_in_i + r_i = \lambda_i \left[\alpha_i Q^{-1} n_i - Q^{-1}r_i \right] \quad \forall 1\le i\le m \\
t_in_i + r_i = (\|r_i\|^2  + \alpha_i t_i)\frac{\alpha_i Q^{-1} n_i - Q^{-1}r_i}{\alpha^2 n_i^TQ^{-1} n_i - r_i^T Q^{-1}r_i} \quad \forall 1\le i\le m \\
\frac{\bar x}{\|\bar x\|} = \frac{\alpha_i Q^{-1} n_i - Q^{-1}r_i}{\alpha^2 n_i^TQ^{-1} n_i - r_i^T Q^{-1}r_i} \quad \forall 1\le i\le m \\
\end{align*}

% \begin{align*}
% t_i = (\|r_i\|^2 + \alpha_i t_i)\frac{\alpha_i n_i^TQ^{-1} n_i - n_i^TQ^{-1}r_i}{\alpha^2 n_i^TQ^{-1} n_i - r_i^T Q^{-1}r_i} \quad \forall 1\le i\le m \\
% \|r\|^2 = (\|r_i\|^2  + \alpha_i t_i)\frac{\alpha_i r_i^TQ^{-1} n_i - r_i^TQ^{-1}r_i}{\alpha^2 n_i^TQ^{-1} n_i - r_i^T Q^{-1}r_i} \quad \forall 1\le i\le m \\
% t_i + \|r_i\|^2 = \|r_i\|^2  + \alpha_i t_i \quad \forall 1\le i\le m \\
% t_i = \|r_i\|^2  \quad \forall 1\le i\le m \\
% t_i = \frac 1 \alpha_i \quad \forall 1\le i\le m \\
% \end{align*}

% We also know that if we let $\bar Q^{-1} =  (n_in_i^T - I)Q^{-1}(n_in_i^T - I)^T$,
% \begin{align*}
% g_i(d^{(i)}) = 0 \\
% \|2 (n_in_i^T - I)d^{(i)}\|^2 = (\alpha_i n_i^Td^{(i)})^2 \\
% \|2 (n_in_i^T - I)\lambda_i Q^{-1}\left((4n_i^Tn_i - 2I)^T\|r_i\| + \alpha_i n_i\right)\|^2 = (\alpha_i n_i^T\lambda_i Q^{-1}\left((4n_i^Tn_i - 2I)^T\|r_i\| + \alpha_i n_i)\right)^2 \\
% \|2 (n_in_i^T - I)\lambda_i Q^{-1}\left((4n_i^Tn_i - 2I)^T\|r_i\| + \alpha_i n_i\right)\|^2 = (\alpha_i n_i^T\lambda_i Q^{-1}\left((4n_i^Tn_i - 2I)^T\|r_i\| + \alpha_i n_i)\right)^2 \\
% A_i^T \lambda_i Q^{-1}A_i = \bar{b_i} \\
% \lambda_i = \frac {\bar{b_i}}{A_i^T  Q^{-1}A_i} \\
% \end{align*}
so that 
\[
d^{(i)} = \lambda_i Q^{-1}A_i = \frac {\bar{b_i}}{A_i^T  Q^{-1}A_i}  Q^{-1}A_i \quad \forall 1\le i\le m.
\]

Because $E \subset P$, we also know that $f(\bar x) \ge 1$ for each $i$. Thus,
\begin{align*}
\frac 1 2 \bar x^{T} Q \bar x \ge 1 \\
\|\bar x \|\bar x^{T} Q \frac{\bar x}{\|\bar x\|} \ge 2 \\
\|\bar x\| \bar x^{T} \frac{\alpha_i n_i - r_i}{\alpha^2 n_i^TQ^{-1} n_i - r_i^T Q^{-1}r_i}\ge 2 \\
\|\bar x\| \bar x^{T} (\alpha_i n_i - r_i)\ge 2\left(\alpha^2 n_i^TQ^{-1} n_i - r_i^T Q^{-1}r_i \right)\\
\|\bar x\| (\alpha_i n_i + r_i) (\alpha_i n_i - r_i)\ge 2\left(\alpha^2 n_i^TQ^{-1} n_i - r_i^T Q^{-1}r_i\right) \\
\|\bar x\| (\alpha_i^2 - \|r_i\|^2)\ge 2\left(\alpha^2 n_i^TQ^{-1} n_i - r_i^T Q^{-1}r_i\right) \\
\alpha_i^2 \|\bar x\| (1 - t^2)\ge 2\left(\alpha^2 n_i^TQ^{-1} n_i - r_i^T Q^{-1}r_i\right) \\
\alpha_i^2 (1 - t^2)\ge 2\bar x(\alpha_i Q^{-1} n_i - Q^{-1}r_i) \\
\alpha_i^2 (1 - t^2)\ge 2(r + \alpha_i n_i)(\alpha_i Q^{-1} n_i - Q^{-1}r_i) \\
\frac 1 2 \alpha_i^2 (1 - t^2)\ge r_i^T(\alpha_i Q^{-1} n_i - Q^{-1}r_i) + \alpha_i n_i^T(\alpha_i Q^{-1} n_i - Q^{-1}r_i)  \\
\frac 1 2 \alpha_i^2 (1 - t^2)\ge \alpha_i^2 n_i^T Q^{-1} n_i - r_i^TQ^{-1}r_i \\
\frac 1 2 (1 - t^2) + \frac{1}{\alpha_i^2}r_i^TQ^{-1}r_i \ge n_i^T Q^{-1} n_i \\
\end{align*}
% \Longrightarrow \frac 1 2 \left(\alpha_i n_i + r_ i\right)^{T} Q (\alpha_i n_i + r_ i) \ge 1 \\
% \Longrightarrow \alpha_i \left(\alpha_i n_i + r_ i\right)^{T} Q n_i + \left(\alpha_i n_i + r_ i\right)^{T} Qr_ i \ge 2 \\
% \Longrightarrow \alpha_i ^2 n_i^{T} Q n_i + \alpha r_ i^{T} Q n_i + \alpha_i n_i^TQr_ i + r_ i^TQr_ i \ge 2 \\
% \Longrightarrow \alpha_i ^2 n_i^{T} Q n_i + 2\alpha r_ i^{T} Q n_i + r_ i^TQr_ i \ge 2 \\

\end{proof}









