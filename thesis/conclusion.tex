This thesis presented several algorithms for solving constrained optimization problems involving black-box functions, where only function values, not derivatives, can be obtained.    Our study, in chapter 3, of the case of linear constraints laid 
a foundation for development of algorithms for more challenging problems involving nonlinear constraints.  
The main theoretical contribution from that study was to show how to construct a well-poised sample set over an ellipsoidal trust region.  We also proposed several methods for constructing this ellipsoidal trust region.

The work of chapter 3 laid a foundation for addressing the case of nonlinear black-box constraints.    A particular challenge for such problems is the fact that we cannot guarantee that infeasible function evaluations will never be attempted; but we strive to minimize such occurences.  Toward that end, we proposed a methodology for constructing second order buffering cones, and we show how to construct an ellipsoidal sample trust region within the intersection of the cones and an outer trust region.   We presented several methods for constructing these ellipsoids and proved that for one of these methods, the resulting ellipsoids are guaranteed to be feasible for sufficiently small outer trust regions, even for nonconvex feasible regions.  

Computational results for our algorithms are encouraging.  However, there are several avenues along which future work is warranted.    First, our convergence results are based upon constructing the sample trust region using a very conservative strategy, which produces ellipsoids that are very small relative to the size of the feasible region intersected with the outer trust region.  We hope to improve upon this by developing provable methods that construct large ellipsoids.

\sbnote{I am blanking out on other ideas for future work, but some more details here would be nice}.




were able to prove under reasonable assumptions that a conservative method for constructing these ellipsoids can ensure that the resulting ellipsoids are guaranteed to be feasible for sufficiently small outer trust regions.  Using this result, we proved that the criticality of the iterates converges to zero.  


%
%Throughout the thesis we have constructed the model-based, derivative-free algorithms \cref{linearly_constrained_dfo_simple} and \cref{constrained_dfo}
%that can run with partially quantifiable constraints.
%This requires constructing a poised, feasible sample set near the current iterate, even when the iterate is nearly infeasible.
%Our construction buffers the feasible region wtih second order cones, and computes a $\sampletrk$ within the buffered region.
%We show that, even for general non-linear constraints, this buffered region is feasible for small $\dk$.
%Also, sample points can be constructed within $\sampletrk$ with only slight modifications to classic model improving algorithms.
%Although our construction has minimal performance alterations, it fairs well to classic derivative free optimization codes,
%and makes fewer infeasible evaluation attempts.
%We also present multiple methods for improving on the sample set, under a framework that ensures convergence.
%
%We present algorithm \cref{modified_model_improving_algorithm} for constructing poised sets over non-ellipsoidal trust regions.
%Future work includes proving corresponding error bounds for sample sets constructed with these polyhedral sample regions.
%Further, we believe that the algorithm presented naturaly faciliates a model improvement algorithm that reduces the number of sample points for thin domains.
%% eliminates unnecessary 
