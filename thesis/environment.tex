%
% File thesis_environment.tex
%
% Description:
% This file contains environments for theorems, lemmas, definitions,
% proofs, corollaries, algorithms, examples, remarks, and assumptions.
% The numbering is set up so that the same counter is used for ALL
% these environments.  This seems to make it easier for readers to
% find referenced material.  The counter is reset for each chapter.
%
% EXAMPLE
%    Definition 3.1
%    Lemma 3.2
%    Example 3.3
%    Theorem 3.4
%    Theorem 3.5
%    Lemma 3.6 ....
%
%    Lemma 4.1
%    Corollary 4.2 ....
%
% The Definition environment is indented on both sides.  This
% is not required, but seemed to be a nice way to separate definitions
% from proven results.
%
%%%%%%%%%%%%%%%%%%%%%%%%%%%%%%%%%%%%



%============================

\newtheorem{theorem}{Theorem}[chapter]
\newtheorem{corollary}[theorem]{Corollary}
\newtheorem{definition}[theorem]{Definition}
\newtheorem{lemma}[theorem]{Lemma}

% \numberwithin{thoerem}{chapter}



\crefname{equation}{}{equations}

\newcounter{assumptioncounter}
\newenvironment{assumption}[1][]{\refstepcounter{assumptioncounter}\par\medskip
\textbf{Assumption \theassumptioncounter} \rmfamily \itshape}{\medskip}

\numberwithin{assumptioncounter}{chapter}
\crefname{assumptioncounter}{Assumption}{assumption}


\newenvironment{boxedcomment}
	{
		\ifbool{showcomments}{
			\par\medskip
			\color{red}%
			\begin{framed}
			\textbf{Comment: }
			\ignorespaces
		}{}
	}
	{
		\ifbool{showcomments}{
			\end{framed}
			\medskip
		}{}
	}
  


\def\proof{\par{\it Proof}. \ignorespaces}

\def\endproof{\vbox{\hrule height0.6pt\hbox{%
   \vrule height1.3ex width0.6pt\hskip0.8ex
   \vrule width0.6pt}\hrule height0.6pt
  }}

  
  
% \newcommand{\new}[1]{{\color{blue}#1}}
% \newcommand\hcancel[2][black]{\setbox0=\hbox{$#2$}\rlap{\raisebox{.45\ht0}{\textcolor{#1} {\rule{\wd0}{1pt}}}}#2} 
% \newcommand{\replace}[2]{{\color{red}\sout{#1}\color{black}{\color{red}#2\color{black}}}} %TeX source markup.
% \newcommand{\replaceb}[2]{{\color{blue}\sout{#1}\color{black}{\color{blue}#2\color{black}}}} %TeX source markup.
% \newcommand{\replacemath}[2]{{\hcancel[red]{#1}{}{\color{red}#2\color{black}}}} %TeX source markup.
% % \newcommand{\replacemath}[2]{{\color{red}#2\color{black}}} %TeX source markup.
% \newcommand{\replacemathb}[2]{{\hcancel[blue]{#1}{}{\color{blue}#2\color{black}}}} %TeX source markup.

\newcommand{\sbnote}[1]{
	\ifbool{showcomments}{
		\textsf{
			{
				\color{cyan}{ SCB note:}   #1
			}
		}
% 		\marginpar{
% 			{
% 				\textbf{Comment}
% 			}
% 		}
	}{}
}

% \newcommand{\sbnote}[1]{
% 		\textsf{
% 			{
% 				\color{cyan}{ SCB note:}   #1
% 			}
% 		}
% 		\marginpar{
% 			{
% 				\textbf{Comment}
% 			}
% 		}
% }
