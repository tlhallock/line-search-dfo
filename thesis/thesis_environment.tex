%
% File thesis_environment.tex
%
% Description:
% This file contains environments for theorems, lemmas, definitions,
% proofs, corollaries, algorithms, examples, remarks, and assumptions.
% The numbering is set up so that the same counter is used for ALL
% these environments.  This seems to make it easier for readers to
% find referenced material.  The counter is reset for each chapter.
%
% EXAMPLE
%    Definition 3.1
%    Lemma 3.2
%    Example 3.3
%    Theorem 3.4
%    Theorem 3.5
%    Lemma 3.6 ....
%
%    Lemma 4.1
%    Corollary 4.2 ....
%
% The Definition environment is indented on both sides.  This
% is not required, but seemed to be a nice way to separate definitions
% from proven results.
%
%%%%%%%%%%%%%%%%%%%%%%%%%%%%%%%%%%%%



\newtheorem{theorem}{Theorem}[chapter]
\newtheorem{lemma}[theorem]{Lemma}
\newtheorem{corollary}[theorem]{Corollary}
\newtheorem{definition}[theorem]{Definition}



\newcounter{assumptioncounter}
\newenvironment{assumption}[1][]{\refstepcounter{assumptioncounter}\par\medskip
\textbf{Assumption \theassumptioncounter} \rmfamily \itshape}{\medskip}

\numberwithin{assumptioncounter}{chapter}
\crefname{assumptioncounter}{Assumption}{assumption}


\newenvironment{boxedcomment}
  {\par\medskip
   \color{red}%
   \begin{framed}
   \textbf{Comment: }\ignorespaces}
 {\end{framed}
  \medskip}
  

\newenvironment{assumptions}
  {\par\medskip
   \color{blue}%
   \begin{framed}
   \textbf{Assumptions: }\ignorespaces}
 {\end{framed}
  \medskip}



\def\proof{\par{\it Proof}. \ignorespaces}

\def\endproof{\vbox{\hrule height0.6pt\hbox{%
   \vrule height1.3ex width0.6pt\hskip0.8ex
   \vrule width0.6pt}\hrule height0.6pt
  }}
